
\documentclass[a4]{article}

\usepackage{graphicx}
\usepackage{fullpage}

\usepackage{code}
\CodeNoNumber

\newenvironment{codecase}[1]{\subsection{#1}}{}


\usepackage[sort]{natbib}
\newcommand{\citeasnoun}{\citet}
\renewcommand{\cite}{\citep}

%\emergencystretch=\hsize
\lefthyphenmin=2
\righthyphenmin=2
%\tolerance=9999

%\setcounter {topnumber}{7}
\renewcommand {\topfraction}{0.99}            % common default: 0.8
\renewcommand {\bottomfraction}{0.99}         % common default: 0.8
\setcounter   {totalnumber}{14}               % common default: 3
\renewcommand{\topfraction}{0.999}
\renewcommand {\textfraction}{0.01}           % common default: 0.2
\renewcommand {\floatpagefraction}{0.99}       % common default: 0.5

\newcommand{\dgrs}{$^{\circ}$}
\newcommand{\twiddle}{\char126}
\newcommand{\pflist}
  {     \renewcommand{\labelitemi}{$\triangleright$}
        \setlength{\itemsep}{0mm}
        \setlength{\parsep}{0mm}
        \setlength{\partopsep}{0mm}
        \setlength{\topsep}{0mm}
        \setlength{\parskip}{0mm}    }

%\newcommand{\usage}[1]{
%  \noindent
%  \begin{tabular}{|l|}\hline
%  Command: {\bf #1}\\
%\hline\end{tabular}\\
%\noindent}

\newenvironment{packed_itemize}{
\begin{itemize}
  \renewcommand{\labelitemi}{$\triangleright$}
  \setlength{\itemsep}{1pt}
  \setlength{\parskip}{0pt}
  \setlength{\parsep}{0pt}
}{\end{itemize}}

\newcommand{\newusage}{\ \\\noindent\makebox[\textwidth]{\hrulefill}}

\newcommand{\usage}[1]{ \begin{packed_itemize} \item {\it Command:} {\tt #1} \end{packed_itemize}}


\newcommand{\ctt}[1]{{\tt #1}}
\newcommand{\packet}[1]{\framebox{#1}}


\title{
YARP Network Protocol Specification 2.0 \\
YARP Companion Utility Specification 2.0 \\
(*** draft ***)
}


%
%  For one-to-one authur/affil correspondence
\author{Paul Fitzpatrick (paulfitz@liralab.it)}


\begin{document}

\maketitle


\begin{abstract}

\noindent
This document specifies YARP network protocol version 2.0,
and a companion utility.  The
protocol and utility is based on code by Giorgio
Metta, Paul Fitzpatrick, Lorenzo Natale, and many others.

\end{abstract}

\pagenumbering{arabic}


\tableofcontents

\section{Introduction}

\subsection{Definition of terms}

The YARP library supports transmission of a stream of
user data across
various protocols -- TCP, UDP, MCAST (multi-cast), shared memory, QNX
message passing -- insulating a user of the library from the
idiosyncratic details of the network technology used.  We call these
these low-level protocols the ``Carriers'', to distinguish them from
the higher-level protocols we will be concerned with here.

For the purposes of YARP, communication takes place throogh
``Connections'' between named entities called ``Ports''.
These form a directed graph, the ``YARP Network'', where Ports are the nodes,
and Connections are the edges.

Each Port is assigned a unique name, such as ``/motor/wheels/left''.  
Every Port is registered by name with
a ``name server''.  The goal is to ensure that if you know the name
of a Port, that is all you need in order to be able to 
communicate with it from any machine.

The purpose of Ports is to move ``Content'' (sequences of bytes representing 
user data) from
one thread to another (or several others) across process and machine
boundaries.  The flow of data can be manipulated and monitored
externally (e.g. from the command-line) at run-time.  In other words,
the edges in the YARP Network are entirely mutable.

Ports are specialized into InputPorts and OutputPorts.
Any OutputPort can send Content to any number of InputPorts.  Any
InputPort can receive Content from any number of OutputPorts.  If an
OutputPort is configured to send Content to an InputPort, they are
said to have a Connection.  Connections can be freely added or 
removed, and may use different Carriers.

The YARP name server is a server that tracks information about ports.
It indexes this information by name, playing a role analogous to
DNS on the internet.
%
To communicate with a port, the properties of that port need to be
known (the machine it is running on, the socket it is listening on,
the carriers it supports).  The YARP name server offers a convenient
place to store these properties, so that only the name of the port is
needed to recover them.
%
The protocol for communicating with the name server
and its operation is specified in section \ref{sect:name-protocol}.

Here are the specifications available in this document:
%
\begin{itemize}

\item Properties of a YARP network.

\item Properties of the YARP utility.

\item Properties of the YARP name server and the protocol used 
for communicating with it.

\item Properties of ports and the protocol used
for communicating with them.

\end{itemize}


\subsection{Properties of a YARP network}

A YARP network consists of the following entities: a set of
ports, a set of connections, a set of names, a name server, and a set
of registrations.

\begin{itemize} \pflist

\item Every port has a unique name.

\item Every connection has a source port and a target port.

\item Each port maintains a list of all connections for which it
is the target port.

\item Each port maintains a list of all connections for which it
is the source port.

\item There is a single name server in a YARP network.

\item The name server maintains a list of registrations.  Each 
registration contains information about a single port, identified
by name.

\end{itemize}


\noindent
Communication within a YARP network can occur between two ports,
between a port and the name server, between a port and an
external entity, and between the name server and an external entity.


\begin{itemize} \pflist

\item Communication between two ports occurs if and only if there
is a connection between them.  That communication uses the
``connection protocol''.

\item Connections involving a port can be created, destroyed, or
queried by communication between an external entity and that port.
This is done by sending ``port commands'' using the YARP
connection protocol.

\item Ports communicate with the name server using the 
``YARP name server protocol''.  Such communication is needed
to create, remove, and query registrations.

\item External entities can also use the YARP name server protocol
to query the name server.

\end{itemize}

\noindent
The standard YARP companion utility can be used to create a
name server, and also to act as an external entity for querying and
modifying the YARP network.

\section{The standard YARP companion utility}

We specify a standard command-line 
utility called ``yarp'' for performing a set of
useful operations for a YARP network.  
%
The functionality described here 
can be provided in other ways also, but
at a minimum this utility should be present on a YARP 2.0 
system.
%
We specify the utility by the user-facing functionality
it provides.  For any of the examples below, the word ``verbose'' 
inserted as the first argument should increase the level of 
detail at which the operation of the utility and problems
encountered is described.


\newusage{}
\usage{yarp}
%
This lists a human-readable summary of the 
ways the utility can be used.  Example output:

\begin{code}
here are ways to use this program:
  <this program> read /name
  <this program> write /name [/target]
  <this program> connect /source /target [carrier]
     carrier can be: tcp udp mcast text
  <this program> disconnect /source /target
  <this program> server
  <this program> where
  <this program> version
  <this program> name {arguments}
\end{code}


\newusage{}
\usage{yarp server}
\usage{yarp server SOCKETPORT}
\usage{yarp server IP SOCKETPORT}
%
This starts a name server running on the current machine, optionally
specifying the socket-port to listen to (default whatever was used in
the previous invocation, as recorded in a configuration file, or 10000
if this is the first time to run).
%
Also, the IP by which the name server should be identified can
optionally be specified (default is a fairly random choice of
the IPs associated with the current machine).

\newusage{}
\usage{yarp where}
%
This will report where the name server is believed to be running,
and the location of the configuration file used to determine that.
Example output:

\begin{code}
Name server is available at ip 5.255.112.225 port 10000
This is configured in file /home/paulfitz/.yarp/conf/namer.conf
You can change the directory where this configuration file is stored
with the YARP\_ROOT environment variable.
\end{code}

\newusage{}
\usage{yarp version}
%
This will report on the yarp version available.  Example:

\begin{code}
YARP network version 2.0
\end{code}

\newusage{}
\usage{yarp name COMMAND $ARG_1$ $ARG_2$ \ldots}
%
This will send the given command and arguments to the name server
using the YARP name server protocol, and report the results.
See Section~\ref{sect:name-protocol}.
%
%Equivalent to the telnet examples given in Section XXX, except the
%``NAME\_SERVER'' prefix is added automatically to the message send to
%the name server.

\newusage{}
\usage{yarp read PORT}
%
This will create an input port of the specified name.  It will
then loop, 
reading ``yarp bottles'' (a simple serialized list) and prints their content
to standard output.  This simple utility is intended for use in testing, or
getting familiar with using YARP.

\newusage{}
\usage{yarp write PORT}
%
This will create an output port of the specified name.  It will then
loop, reading from standard input and writing yarp bottles..
Optionally, a list of input ports to connect to automatically can be
appended to the command.  This simple utility is intended for use in
testing, or getting familiar with using YARP.

\newusage{}
\usage{yarp connect OUTPUT\_PORT INPUT\_PORT}
\usage{yarp connect OUTPUT\_PORT INPUT\_PORT CARRIER}
%
This will request the specified output port to send its output in 
future to the specified input port.
Optionally, the carrier to be used can be added as an extra argument
(e.g. tcp, udp, mcast, ...).

\newusage{}
\usage{yarp disconnect OUTPUT\_PORT INPUT\_PORT}
%
This will request the specified output port to cease sending its output to
the specified input port.

Appendix \ref{sect:using-utility} has a user guide to getting started
with the YARP network companion utility.


\section{The connection protocol}

This is the protocol used for a single connection from
an output port to an input port.  We discuss how this
process gets initiated in the next section.
At the point of creation of a connection, the
following information is needed:

\begin{itemize}

\item An address -- the machine name and socket-port at which
the input port is listening.

\item The name of the input port.

\item The name of the output port associated with the
connection.  This name
needs to be retained for proper disconnection in some cases.
If the connection is not actually associated with a port,
but is initiated by an external entity, then the name is
not important and should be set to ``external'' (or any
name without a leading slash characher).

\end{itemize}

\subsection{Basic phases}

The connection protocol has several phases -- header,
index, and body.

\begin{itemize}

\item Initiation phase
  \begin{itemize}
  \item We begin once the sender has
    successfully opened a tcp socket connection to the receiver
    (assuming that is the carrier it is registered with).
  \end{itemize}

\item Header phase
  \begin{itemize}
  \item This phase follows immediately after the initiation phase.
  \item Transmission of protocol specifier
    \begin{itemize}
      \item Sender transmits 8 bytes that identify the carrier
	that will be used.  The header may be used to pass a 
	few flags also.
      \item Receiver expects 8 bytes, and attempts to find a
	carrier that is consistent with them.
    \end{itemize}
  \item Transmission of sender name
    \begin{itemize}
      \item Sender transmits the name of the output port
	it is associated with, in a carrier specific way.
      \item Receiver expects the name of the output port,
	transmitted in a carrier specific way.
    \end{itemize}
  \item Transmission of extra header material
    \begin{itemize}
      \item Sender may transmit extra information, depending on
      the carrier.
      \item Receiver may expect extra information, depending on
      the carrier.
    \end{itemize}
  \end{itemize}

\item Header reply phase

  \begin{itemize}
    
  \item This phase follows immediately after the header phase, and
    concludes the preamble to actual data transmission.  After this
    phase, the two ports are considered connected.
    
  \item Receiver may transmit some data, depending on the carrier.
    Receiver then may switch from the initial network
    protocol used to something else (udp, mcast, etc), again
    depending on the carrier.
    
  \item Sender may expect some data, depending on the carrier.
    Sender then may switch from the initial network
    protocol used to something else (udp, mcast, etc), again
    depending on the carrier.

  \end{itemize}

\item Index phase
    
  \begin{itemize}
    
  \item Sender sends carrier-dependent data describing properties of
    the payload data to come.
    
  \item Receiver expects carrier-dependent data describing properties
    of the payload data to come.
    
  \end{itemize}
  
\item Payload data phase
  
  \begin{itemize}
    
  \item Sender sends carrier-dependent expression of user data (maybe none).
    
  \item Receiver expects carrier-dependent expression of user data.
    
  \end{itemize}

\item Acknowledgement phase

  \begin{itemize}
    
  \item Receiver sends carrier-dependent acknowledgement of receipt of
    payload data (maybe none).
    
  \item Sender expects carrier-dependent acknowledgement of receipt of
    payload data (maybe none).
    
  \end{itemize}
  
\end{itemize}


\noindent
This is the basic pattern of YARP communication between ports.
Clearly different carriers have a lot of freedom in how they operate.


\subsection{The ``tcp'' carrier}


\begin{itemize}

\item{Header and header reply}

  \begin{itemize}
    
  \item The 8-byte protocol specifier for tcp is: \packet{'Y' 'A' 0xE4
    0x1E 0 0 'R' 'P'}.  Another possible variant is: \packet{'Y' 'A'
    0x64 0x1E 0 0 'R' 'P'}.  The first version identifies a connection
    that sends acknowledgements; the second is for connections that
    omit acknowledgements.
    
  \item The sender name is transmitted and expected to be in the
    following format: a 4 byte integer (little endian) giving the length
    of port, followed by the port name in characters, followed by the null
    character.

  \item There is no extra header material for this carrier.

  \item The header reply is as 8 bytes long: \packet{'Y' 'A' B1 B2 0 0
    'R' 'P'}, where (B1,B2) is a (little-endian) two-byte integer
    specifying a socket-port number (unused).  

  \item After the header reply, there is no switch in network protocol
    -- the initial tcp connection continues to be used.

    \end{itemize}

\item{Index, payload, and acknowledgement}

  \begin{itemize}

  \item The sender transmits 8 bytes: \packet{'Y' 'A' 10 0 0 0 'R' 'P'}.
    This identifies the length of the ``index header'' as 10.
    
  \item The sender transmits 10 bytes: \packet{LEN 1 255 255 255 255
    255 255 255 255}.  LEN is the number of blocks of user data need
    to be sent.  This byte-sequence says there are LEN send blocks, 1
    reply block expected, and that the sizes will be listed
    individually next (this odd format is for backward compatability
    with older YARP versions).
      
  \item The sender transmits LEN 4-byte little-endian integers, one
    for each of the LEN blocks of user data, giving the length of each
    block.
      
  \item The sender transmits 4 bytes: \packet{0 0 0 0}. This asks for
    a reply length of 0.

  \item If this is the variant of the tcp carrier that requires
  acknowledgments, then the receiver sends 8 bytes: \packet{'Y' 'A' B1
  B2 B3 B4 'R' 'P'}, where B1-4 is a little-endian integer giving a
  length (could be 0).  It then sends that number of extra bytes.

  \end{itemize}

\end{itemize}


\subsection{The ``udp'' carrier}


\begin{itemize}

\item{Header and header reply}

  \begin{itemize}
    
  \item The 8-byte protocol specifier for udp is: \packet{'Y' 'A' 0x61
    0x1E 0 0 'R' 'P'}.  The following variant of this should also be
    accepted: \packet{'Y' 'A' 0xE1 0x1E 0 0 'R' 'P'} (it is the same
    thing).

  \item Otherwise header and header reply are identical to the tcp case.
  \item After the header reply, both sides switch to a udp connection
    to the socket-port specified in the header reply.

  \end{itemize}

\item{Index, payload, and acknowledgement}

  \begin{itemize}

  \item Identical to tcp.  Data is split arbitrarily to fit into
    datagrams.
    
  \item Acknowledgments are not a possibility.

  \end{itemize}

\end{itemize}


\subsection{The ``mcast'' carrier}

\begin{itemize}

\item{Header and header reply}

  \begin{itemize}
    
  \item The 8-byte protocol specifier for mcast is: \packet{'Y' 'A' 0x62
    0x1E 0 0 'R' 'P'}.  The following variant of this should also be
    accepted: \packet{'Y' 'A' 0xE2 0x1E 0 0 'R' 'P'} (it is the same
    thing).

  \item The sender name is sent as for tcp.

  \item Extra header material is send -- 6 bytes.  The first 4 bytes
  specify a multicast IP address.  Next 2 bytes are a (bigendian)
  integer giving a socket-port number.  Note that producing these
  numbers can be helped by side communication with the name server.

  \item There is no header reply for mcast.
  
  \item Both sides switch to a multi-cast group on the specified IP
  and socket-port.

  \end{itemize}

\item{Index, payload, and acknowledgement}

  \begin{itemize}

  \item Identical to udp.

  \item But at most one connection from a given port with an mcast carrier
    should actually write to the multi-cast group.

  \end{itemize}
  
\end{itemize}



\subsection{The ``text'' carrier}

\begin{itemize}

\item This carrier is carefully designed to make it easy to type into
  a terminal.

\item{Header and header reply}

  \begin{itemize}
    
  \item The 8-byte protocol specifier for text is: \packet{'C' 'O' 'N' 'N' 'E' 'C' 'T' '\textvisiblespace{}'}.  

  \item The sender name is sent as plain text followed by the newline
  character `$\backslash$n'.

  \item There is no extra material

  \item There is no header reply expected for text.
  
  \item There is no network protocol switch.

  \end{itemize}

\item{Index, payload, and acknowledgement}

  \begin{itemize}

  \item There is no index.

  \item The payload is expected to be a series of lines of text
    terminated by the newline character `$\backslash$n'.

  \item There is no acknowledgement expected for text.

  \end{itemize}
  
\end{itemize}


\subsection{The ``shmem'' carrier}

This is essentially the same as the tcp carrier, except that there is no
header reply, and there is a shift in protocol after header transmission
on both sides to an ACE shared memory stream.  This carrier is 
currently being reworked to make its specification independent of ACE, 
and to further improve efficiency in an existing implementation.

The advantage of this carrier is that it is fast -- the best way to
send messages between processes on a single machine.  Of course, it
doesn't work for processes on different machines.


\subsection{The ``local'' carrier}

This is a new carrier designed specifically for communication between
threads in a single process.  Giving a specification for the protocol
it uses has low priority, since two such threads are unlikely to be
using different YARP implementations.


\subsection{Known protocol specifiers}

Here are the currently known protocol specifiers.
The ``shmem'' carrier is not yet documented, but is
implemented in the C++ version of YARP.

\begin{figure}[h]
\begin{tabular}{|cccccccc|l|l|}
\hline
\multicolumn{8}{|c|}{\bf 8-byte magic number} & {\bf protocol} & {\bf variant} \\ \hline\hline
`Y' & `A' & 0x61 & 0x1E & 0 & 0 & `R' & `P'  & udp & \\
`Y' & `A' & 0xE1 & 0x1E & 0 & 0 & `R' & `P'  & udp & \\
`Y' & `A' & 0x62 & 0x1E & 0 & 0 & `R' & `P'  & mcast & \\
`Y' & `A' & 0xE2 & 0x1E & 0 & 0 & `R' & `P'  & mcast & \\
`Y' & `A' & 0x63 & 0x1E & 0 & 0 & `R' & `P'  & shmem & \\
`Y' & `A' & 0xE3 & 0x1E & 0 & 0 & `R' & `P'  & shmem & \\
`Y' & `A' & 0x64 & 0x1E & 0 & 0 & `R' & `P'  & tcp & without acks \\
`Y' & `A' & 0xE4 & 0x1E & 0 & 0 & `R' & `P'  & tcp & with acks \\
`C' & `O' & `N'  & `N'  & `E' & `C' & `T' & `\textvisiblespace{}'  & text & \\
`L' & `O' & `C'  & `A'  & `L' & `I' & `T' & Y  & local & \\
\hline
\end{tabular}
\end{figure}

\section{Port commands}

Every port is always available for new connections from external
entities -- to request that new connections between ports be created,
old connections be removed, to inquire after status, etc.
%
The protocol used for communicating with a port is layered on top of the
protocol described in the previous section.  Any carrier can be used.
The ``payload data'' is as follows:

\begin{itemize}

\item We send an 8 byte header: \packet{0,0,0,0, `\twiddle', CHAR, 0, 1.}

\item CHAR is a character that identifies what the message is about.

  \begin{itemize}
    
  \item CHAR = `d': this header is used to signal that user data is
  arriving next, as opposed to a port command
    
  \item CHAR = anything else: this signals that a port command
    follows.

  \end{itemize}

\item for the port command case (CHAR = 0) the remainder of the message 
  is interpreted as a string S.

  \begin{itemize}
   \item S begins with `/', e.g. `/read': this is a request to add a
   Connection to the named InputPort.

   \item S begins with `!', e.g. `!/read': this is a request to remove
   a Connection to the named InputPort.

   \item S begins with `\twiddle', e.g. `\twiddle/read': this is a
   request to remove a Connection from the named OutputPort.

   \item S is `*': this is a request for the port to dump information
     about what it is connected to.

   \item S is `q': the specific connection that the command is received on
   should now shut down.
  \end{itemize}

\end{itemize}

\noindent
Alternatively, with the ``text'' carrier, we send a string terminated
in `$\backslash$n'.  This is the string S.  The first letter is copied
to be CHAR.



\subsection{YARP URIs}

Port names in YARP can contain multiple special elements.
We've seen names such as ``/write''.  We can also have
names such as ``udp://write'' which means ``connect
to the port named /write using the udp carrier''.

We can also prepend a network selector of the form
``/net=NETNAME/''.  For example, a name such as
``udp://net=196/write'' means ``connect to the 
port named /write using the udp carrier, and make the
connection on the network with ip addresses beginning with
196''.
%
This is useful in scenarios with multiple networks, where it may be
desirable to route connections through particular networks (for
example, to devote a network to time-critical traffic).  This
functionality is supported primarily with the help of the name server.
The ip it reports for a machine is usually a reasonable default, but
the user can choose using ``net='' to request a name on a particular
network.

Symbolic network names can be configured.  This process is not yet
specified.  You can do it right now by setting properties of a fake
port called ``networks'' (no leading slash), where the properties are
symbolic names and their values are the numeric network IP prefix.
But this process will change.

%% How should multiple networks work?  Currently in YARP1, a port is
%% associated with a particular network.  If two ports are on different
%% networks, they can't communicate, even if there is connectivity
%% (I think).

%% It would seem more desirable to associate the choice of network,
%% if there is a choice, with the edges (connections) and not the
%% nodes (ports).

%% Let's require that all nodes can reach the name server and vice
%% versa.

%% Single-hop network choice for tcp and udp can be made by selecting one
%% of a choice of IPs for the target.  Could add a connection attribute
%% ``over 192.*'' as a way to choose net?  Or text names.

%% Done!


\subsection{Carriers supported}

An implementation of YARP2 must support at least the ``tcp'' carrier.
Other carriers that may be supported:
``text'', ``udp'', ``mcast'', ``shmem'', ``qnx'', ``local''.

As a place to start an implementation, the ``text'' carrier is very
simple to implement, and can masquerade as ``tcp'' for the purposes
of initial handshaking.

To see this, get the ``netcat'' program (available as debian package
of the same name).  In one terminal, run:
%
\begin{code}
nc -l -p 9000
\end{code}
%
This starts a tcp listener on socket-port 9000, and prints out
any data that arrives there.  Then tell the name server to
create an entry for this listener, and tag it as accepting text:
%
\begin{code}
yarp name register /nc tcp ... 9000
yarp name set /nc accepts text
\end{code}
%
Now lets write some data to that port.
%
\begin{code}
yarp write /write text://nc
\end{code}
%
Type something in, such as ``hello world'', and hit return.
On the terminal running {\tt nc} (netcat) you should see:
%
\begin{code}
CONNECT /write
d
0 "hello world"
\end{code}
%
This is what text mode looks like, for the particular
data type used by yarp read and write (``bottles'').  
As we saw in an earlier section, we can also write to ports in text mode.
And if we were to restart nc and then try the following:
%
\begin{code}
yarp connect text://nc /foo
\end{code}
%
\begin{code}
CONNECT external
/foo
\end{code}
%
This is what a command to connect looks like in YARP2.  If we omit the
``text:/'' then the tcp carrier may be used, which is compatible with YARP1
but is a bit less trivial to work with.  Once our YARP implementations
are up to date, the default command carrier will be switched to text.



\section{The name server protocol}

\label{sect:name-protocol}

The name server is a program that listens on a known socket-port\footnote{We
write socket-ports to distinuish tcp/ip port numbers for sockets, 
from higher-level YARP Ports.}
on a known machine.
%
It tracks information about Ports in the YARP Network.
%
%
If you know the name
of a Port, you can query the name server to learn how to communicate
with that Port.

The name server maintains a set of records, whose key is a text
string.  The contents are at least hostname, socket number, and
protocol name.  This describes how to contact the port.  There is also
a description of what kinds of connections the port can or is willing
to participate in.  The set of protocols the port can accept an
incoming data connection for are named - this is the ``accept'' set.
The set of protocols the port can create an outgoing data connection
for are also named - this is the ``offer'' set.

For example, suppose you want to communicate with a Port called
``/write''.  The first step is to ask the name server about this
Port.  The name server runs on a known socket-port of a known machine,
listening for tcp connections.
It is usually queried through a library call, but for illustration
purposes we describe
querying it using telnet.  Suppose the name server is running on
machine 192.168.0.3 and listening on socket-port 10000 (we will
discuss a procedure for discovering this information later).
Then we can query the name server about the Port /write as follows:


\begin{code}
telnet 192.168.0.3 10000
\end{code}
%
The name server should start listening -- if the connection is refused,
something is wrong.  Once the connection is made, type:
%
\begin{code}
NAME_SERVER query /write
\end{code}
%
The server will respond with something of the form:
%
\begin{code}
registration name /write ip 5.255.112.227 port 10001 type tcp
*** end of message
\end{code}
%
So the Port named /write is listening on the machine with
IP address 5.255.112.227, on port 10001, and it expects TCP
connections.

How do Ports get registered in the same place?  Here's how to create a
(fake) registration
manually (usually it is of course done through a library call).
Telnet to the name server as before, and type:
%
\begin{code}
NAME_SERVER register /write
\end{code}
%
The server will respond with something of the form:
%
\begin{code}
registration name /write ip 5.255.112.227 port 10001 type tcp
*** end of message
\end{code}

The name server takes responsibility for allocating
socket-ports and identifying the machine the Port runs on.

Note that the protocol described here for communicating with the name
server is a YARP2 feature.  YARP1 used a different, binary protocol.
The human-readable protocol has been introduced to make the system
more transparent and easier to step through.

For yarp utilities to correctly discover how
to contact the name server,
there should be a file namer.conf in the directory \$HOME/.yarp/conf/
(or in the directory specified by an environment variable \$YARP\_ROOT)
that looks like this:
\begin{code}
192.168.0.3 10000
\end{code}
This gives the machine and socket-port that the name server is assumed
to be running on.

If this file does not exist, or is incorrect, yarp utilities will
attempt to contact the nameserver using multi-cast broadcasts to
224.2.1.1 port 10001 (this is a YARP2 feature, not available in
YARP1).  If the nameserver is running a machine reachable from
multi-cast, it will respond with its ``true'' tcp address, which will
then be used by the utility.  The configuration file will be updated
automatically for future reference.  The multi-cast protocol is
identical to the normal tcp protocol.  Clients can broadcast
``NAME\_SERVER query root'' to trigger the name server to send a record
of the form ``registration name root ip ADDRESS port NUMBER type
CARRIER''.  The ``root'' record is a special record corresponding to
the name server.  Multi-cast broadcasts should not generally be used
by clients to communicate with the name server, since the output of
the name server is not tagged with the recipient, so there is the
potential for cross-talk.


\newusage{}
\usage{NAME\_SERVER query PORT}
Requests registration information for the named port.  Response is of 
the following form:
\begin{code}
registration name PORT ip ADDRESS port NUMBER type CARRIER
*** end of message
\end{code}
For example:
\begin{code}
registration name /write ip 5.255.112.227 port 10001 type tcp
*** end of message
\end{code}
If there is no registration for the port, the registration line
is omitted, and instead the response is simply:
\begin{code}
*** end of message
\end{code}


\newusage{}
\usage{NAME\_SERVER register PORT}

Requests creation of registration information for the named port.  
Response is of the following form:
\begin{code}
registration name PORT ip ADDRESS port NUMBER type CARRIER
*** end of message
\end{code}
For example:
\begin{code}
registration name /write ip 5.255.112.227 port 10001 type tcp
*** end of message
\end{code}
%
Optionally, the user can take responsibility for more, and 
issue commands in one of the following forms:
\begin{code}
NAME_SERVER register PORT CARRIER
NAME_SERVER register PORT CARRIER IP
NAME_SERVER register PORT CARRIER IP NUMBER
\end{code}
Any value (including the port name) can be replaced by ``...'' to leave it 
up to the name-server to choose it.  For example:
\begin{code}
NAME_SERVER register ... tcp 127.0.0.1 8080
\end{code}
Gives something of the form:
\begin{code}
registration name /tmp/port/1 ip 127.0.0.1 port 8080 type tcp
*** end of message
\end{code}
If you choose to set the ip yourself, be careful -- there is the 
possibility of problems with multiple ways to identify the same
machine.  It is best to let the name server choose a name,
which it should do in a consistent way.  If a machine has
multiple ip addresses on multiple networks, that can be 
handled -- see the 
discussion of the {\tt ips} property in the section on {\tt set}.
That is important for the purposes of controlling which 
network is used for connections from one port to another.




\newusage{}
\usage{NAME\_SERVER unregister PORT}

Removes registration information for the named port.  
Response is of the following form:
\begin{code}
*** end of message
\end{code}


\newusage{}
\usage{NAME\_SERVER list}

Gives registration information of all known ports.
Response is of the following form:
\begin{code}
registration name /write ip 130.251.4.159 port 10021 type tcp
registration name /read ip 130.251.4.159 port 10031 type tcp
registration name /tmp/port/4 ip 130.251.4.159 port 10011 type tcp
registration name /tmp/port/3 ip 130.251.4.52 port 10021 type tcp
registration name /tmp/port/2 ip 130.251.4.52 port 10011 type tcp
registration name /tmp/port/1 ip 130.251.4.159 port 10001 type tcp
*** end of message
\end{code}



\newusage{}
\usage{NAME\_SERVER set PORT PROPERTY VALUE1 VALUE2 \ldots}

The name server can store extra properties of a port, beyond the
bare details associated with registration.  The {\tt set} command
is used to do this.  For example, the command:
\begin{code}
NAME_SERVER set /write offers tcp udp mcast
\end{code}
Gets the following response:
\begin{code}
port /write property offers = tcp udp mcast
\end{code}
The {\tt get} and {\tt check} commands can then be used to query
such properties.

There are some special properties used by YARP.  Property ``ips''
can list multiple identifiers of a machine.  Property ``offers''
lists carriers that an output port can support.  Propery ``accepts''
lists carriers that an input port can support.




\newusage
\usage{NAME\_SERVER get PORT PROPERTY}
Gets the values of a stored property. For example, 
after the {\tt set} command example shown earlier, the command:
\begin{code}
NAME_SERVER get /write offers
\end{code}
Returns the following response:
\begin{code}
port /write property offers = tcp udp mcast
\end{code}

\newusage
\usage{NAME\_SERVER check PORT PROPERTY VALUE}
Checks if a stored property can take the given value. For example, 
after the {\tt set} command example shown earlier, the command:
\begin{code}
NAME_SERVER check /write offers tcp
\end{code}
Returns the following response:
\begin{code}
port /write property offers value tcp present true
\end{code}

\newusage
\usage{NAME\_SERVER route PORT1 PORT2}
Finds a good way to connect an output port to an input port, based
on the carriers they have in common (preferred carriers can optionally
be added to this command in decreasing order of preference) and
which carriers are physically possible (for example, `shmem'
requires ports to be on the same machine, and `local' requires
ports to belong to threads with a shared memory space).
For example, the command:
\begin{code}
NAME_SERVER route /write /read
\end{code}
Returns the following response:
\begin{code}
port /write route /read = shmem://read
\end{code}
Suggesting that shmem is the best carrier to use.


%% \begin{codecase}{NAME\_SERVER check PORT PROPERTY VALUE}
%% Not yet documented.
%% \end{codecase}

%% \begin{codecase}{NAME\_SERVER match PORT PROPERTY VALUE}
%% Not yet documented.
%% \end{codecase}

%% \begin{codecase}{NAME\_SERVER connect}
%% Not yet documented.
%% \end{codecase}

%% \begin{codecase}{quit}
%% Not yet documented.
%% \end{codecase}




\section{Status of implementations}


\subsection{C++}

There is a partially-conforming implementation of YARP2 written
in C++.  It is available in CVS from:
\begin{code}
  CVSROOT = cvs.sourceforge.net:/cvsroot/yarp0
  Modules = yarp/src/libYARP_OS yarp/src/tools yarp/conf
\end{code}
See yarp0.sourceforge.net for instructions on how to access this.
This code is mature, well-tested, but does not yet conform fully
with the protocols described in this document.

There is a drop-in replacement implementation of YARP2 being developed
in:

\begin{code}
  Modules = yarp/src/libYARP_OS2
\end{code}

\indent
That implementation follows this document carefully, but at the
time of writing is not yet tested much in the real world.

\subsection{Matlab}

There is a Matlab wrapper around the C++ implementation --
see {\tt yarp/src/libYARP\_matlab} in the above repository.

\subsection{Python}

Python wrappers around the C++ implementation exist.


\subsection{Java}

There is a conforming implementation of YARP2 written in Java.
It is available in CVS from in the {\tt yarp/src/java}
directory.
See yarp0.sourceforge.net for instructions on how to access this.


\appendix

\section{Using the YARP companion utility}

\label{sect:using-utility}

This is informal, tutorial-style section meant to give the flavor
of using YARP.

First, please
open two terminal windows; we'll refer to these as A and B.
%
Before doing any communication, is is necessary to start the YARP name
service -- this is a program which keeps track of what YARP resources
are currently available and how to access them.
%
In terminal A, type:

\begin{verbatim}
yarp server
\end{verbatim}

If all goes well, you will see the message ``yarp: name server starting at ...'' -- if not, check any configuration file
mentioned in the output of ``yarp where''.
%
Next, in terminal B, type:
%
\begin{verbatim}
yarp check
\end{verbatim}
%
This will try to communicate with the name server, send a message,
receive a message, and basically make sure that everything is
working well.  Here is what you should see:

\begin{quote}
{\small
\begin{verbatim}
==================================================================
=== Trying to register some ports
yarp: Registered port ... as tcp://5.255.112.225:10011
yarp: Registered port ... as tcp://5.255.112.225:10001
==================================================================
=== Trying to connect some ports
yarp: /tmp/port/2: Receiving input from /tmp/port/1 to /tmp/port/2 using tcp
yarp: Sending output from /tmp/port/1 to /tmp/port/2 using tcp
==================================================================
=== Trying to write some data
==================================================================
=== Trying to read some data
*** Read number 42
==================================================================
=== Trying to close some ports
yarp: Shutting down port /tmp/port/2
yarp: /tmp/port/2: Stopping input from /tmp/port/1 to /tmp/port/2
yarp: stopping output from /tmp/port/1 to /tmp/port/2
yarp: Shutting down port /tmp/port/1
*** YARP seems okay!
\end{verbatim}
}
\end{quote}
%
\noindent
Here is an example of things going wrong:
%
\begin{quote}
{\small
\begin{verbatim}
==================================================================
=== Trying to register some ports
yarp: Couldn't connect to 5.255.112.225
yarp: Name server missing
\end{verbatim}
}
\end{quote}

If networking is broken on your machine, then YARP will have trouble.
If you are running the YARP name server on a machine that does not
have a static IP address or a name registered with DNS, then you
should check the configuration file mentioned by ``yarp where'', and
make sure that the first line of the configuration file is of the form:
\begin{quote}
  current.valid.ip.address 10000
\end{quote}

The number ``10000'' is the default port number that the YARP name
service uses.  This is usually fine; if you have a conflict with this,
then please change it in the configuration file.

If all is well, we can try exercising YARP a little more.

In terminal B, type:
%
\begin{verbatim}
yarp read /port1
\end{verbatim}
%
This will report \ctt{yarp: Registered port /port1 ...} in terminal B and in
terminal A some corresponding message will also appear.
%
Next, in terminal C, type:

\begin{verbatim}
yarp write /port2
\end{verbatim}

If all goes well, this will report \ctt{yarp: Registered port /port2 ...} in
terminal C and in terminal A some corresponding message will also
appear.
%
Now, in terminal D, type:

\begin{verbatim}
yarp connect /port2 /port1
\end{verbatim}

Terminal C will report \ctt{Connecting /port2 to /port1}.  Now if
you type ``hello world'' into terminal C, and hit enter, that text
will appear on terminal B.


If everything is going okay so far, type on terminal E:
%
\begin{verbatim}
yarp read /port3
\end{verbatim}
%
and on terminal D, type:
%
\begin{verbatim}
yarp connect /port2 /port3
\end{verbatim}
%
Now any line you enter in terminal C should appear in both terminal B and E.

\section{Manually interacting with ports}

Suppose we have created ports as follows by typing the
following in different terminals:
\begin{code}
  yarp server
  yarp write /write
  yarp read /read
  yarp read /read2
\end{code}
%
We could connect and disconnect ports using the YARP companion
utility, but here's how we could do the same thing ``manually'':

\begin{code}
command:  yarp where
response: Name server is available at ip 192.168.0.3 port 10000

command:  telnet 192.168.0.3 10000
type:     NAME_SERVER query /write
response: registration name /write ip 192.168.0.3 port 10001 type tcp
          *** end of message
          [connection closes]

command:  telnet 192.168.0.3 10001
type:     CONNECT anonymous
response: Welcome anonymous
type:     *
response: This is /write
          There are no outgoing connections
          There is this connection from anonymous to /write using protocol tcp
          *** end of message
type:     /read
response: Connected to /read
type:     *
response: This is /write
          There is a connection from /write to /read using protocol tcp
          There is this connection from anonymous to /write using protocol tcp
          *** end of message
type:     !/read
response: Removing connection from /write to /read
type:     /mcast://read
response: Connected to /read
type:     /read2
response: Connected to /read2
type:     *
response: This is /write
          There is a connection from /write to /read using protocol mcast
          There is a connection from /write to /read2 using protocol tcp
          There is this connection from anonymous to /write using protocol tcp
          *** end of message
type:     q
response: Bye bye
          [connection closes]

command:  telnet 192.168.0.3 10000
type:     NAME_SERVER query /read
response: registration name /write ip 192.168.0.3 port 10002 type tcp
          *** end of message
          [connection closes]

command:  telnet 192.168.0.3 10002
type:     CONNECT anonymous
response: Welcome anonymous
type:     *
response: This is /read
          There are no outgoing connections
          There is a connection from /write to /read using protocol mcast
          There is this connection from anonymous to /read using protocol tcp
          *** end of message
type:     q
response: Bye bye
          [connection closes]

\end{code}





\end{document}
