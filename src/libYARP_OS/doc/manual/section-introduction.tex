
\section{What is YARP?}

YARP is written by and for researchers in robotics, particularly
humanoid robotics, who find themselves with a complicated pile of
hardware to control with an equally complicated pile of software.  At
the time of writing (2004), running decent visual, auditory, and
tactile perception while performing elaborate motor control in
real-time requires a lot of computation.  The easiest and most
scalable way to do this right now is to have a cluster of computers.
Every year what one machine can do grows, but so do our demands.  YARP
is a set of tools we have found useful for meeting our computational
needs for controlling various humanoid robots.

The components of YARP can be broken down into:

\begin{itemize}

\item {\bf libYARP\_OS} -- interfacing with the operating system(s)
to support easy streaming of data across many threads across many
machines.  YARP is written to be OS neutral, and explicitly supports
Linux, Microsoft Windows, and the QNX realtime operating system.

\item {\bf libYARP\_dev} -- interfacing with common devices used in robotics:
framegrabbers, digital cameras, motor control boards, etc.

\item {\bf libYARP\_sig} -- performing common signal processing tasks
(visual, auditory) in an open manner easily interfaced with other
commonly used libraries.

\item {\bf ...} --- more documentation to come.

\end{itemize}

These components are maintained separately.  The core component
is {\bf libYARP\_OS}, which must be available before the other 
components can be used.  This manual describes how to install and
use that component first, then the others.



\section{Requirements}

YARP is tested on Windows (2000/XP), Linux (Debian/SuSE), and QNX6.
It is based on the open-source ACE (ADAPTIVE Communication
Environment) library, which is portable across a very broad range of
environments, and YARP should inherit that portability.

For real-time operation, network overhead has to be minimized, so YARP
is designed to operate on an isolated network or behind a firewall.
If you expose machines running YARP to the internet, expect your robot
to one day be commanded to make a crude gesture at your funders by a
script kiddie in New Zealand (or, if you are in New Zealand, New
York).

For interfacing with hardware, we are at the mercy of which operating
systems particular companies choose to support~-- few are enlightened
enough to provide source.  The {\bf libYARP\_dev} library is structured
to interface easily with vendor-supplied code, but to shield the rest
of your system from that code so that future hardware replacements
are possible.  Check the requirements imposed by your current hardware;
YARP will not reduce these, only make future changes easier.


