
\section{What is YARP?}

YARP is written by and for researchers in robotics, particularly
humanoid robotics, who find themselves with a complicated pile of
hardware to control with an equally complicated pile of software. 
\marginpar{YARP was developed initially by Paul and Giorgo at MIT 
based on a core of classes developed by Paul and used extensively 
for the first time on COG running QNX4.25}
At the time of writing (2005), running decent visual, auditory, and
tactile perception while performing elaborate motor control in
real-time requires a lot of computation.  The easiest and most
scalable way to do this right now is to have a cluster of computers.
Every year what one machine can do grows, but so do our demands.  YARP
is a set of tools we have found useful for meeting our computational
needs for controlling various humanoid robots. 

The components of YARP can be broken down into:

\begin{itemize}

\item {\bf libYARP\_OS} -- interfacing with the operating system(s)
to support easy streaming of data across many threads across many
machines.  YARP is written to be OS neutral, and explicitly supports
Linux, Microsoft Windows, Mac Os (aka Darwin) and the QNX realtime operating system.

\item {\bf libYARP\_sig} -- performing common signal processing tasks
(visual, auditory) in an open manner easily interfaced with other
commonly used libraries.

\item {\bf libYARP\_sig\_logpolar} -- performing the logpolar subsampling of images. This is the author's preference when doing image processing for robots. It offers full field of view with reduced number of pixels to process. For a full reference please check the relative manual.

\item {\bf libYARP\_dev} -- interfacing with common devices used in robotics:
framegrabbers, digital cameras, motor control boards, etc.

\item {\bf libYARP\_robot} -- common interface to different robotic platform with easy to replace device drivers as in the {\bf libYARP\_dev} library.

\item {\bf tools} -- various support tools, utilities, and executable components.

\end{itemize}

These components are maintained separately.  The core component
is {\bf libYARP\_OS}, which must be available before the other 
components can be used.  This manual describes how to install and
use that component first, then the others.



\subsection{Requirements}

YARP is tested on Windows (2000/XP), Linux (Debian/SuSE), MacOs (Darwin), and QNX6.
It is based on the open-source ACE (ADAPTIVE Communication
Environment) library, which is portable across a very broad range of
environments, and YARP should inherit that portability. YARP is written almost 
entirely in C++.

For real-time operation, network overhead has to be minimized, so YARP
is designed to operate on an isolated network or behind a firewall.
If you expose machines running YARP to the internet, expect your robot
to one day be commanded to make a crude gesture at your funders by a
script kiddie in New Zealand (or, if you are in New Zealand, New
York).

For interfacing with hardware, we are at the mercy of which operating
systems particular companies choose to support~-- few are enlightened
enough to provide source.  The {\bf libYARP\_dev} library is structured
to interface easily with vendor-supplied code, but to shield the rest
of your system from that code so that future hardware replacements
are possible.  Check the requirements imposed by your current hardware;
YARP will not reduce these, only make future changes easier.

With the same spirit the {\bf libYARP\_robot} tries to provide a common interface to different robotic platforms. Still the idea is that of shielding higher level code from the nitty-gritty details of the robot hardware and configuration. Porting to a new platform is not guaranteed to be anything trivial but for the reuse of higher level code. The amount of recycling clearly depends on the differences among control cards, frame grabbers, etc. Unfortunately, this doesn't shield you from knowing the platform you're going to work on, but we bet you need to know it anyway.


YARP has consequently three levels of configuration: operating system, hardware, and robot level. The first level of configuration should concern you only if you're planning to compile YARP on a new operating system. There are a few places where to intervene that we chose not to relay to ACE (for now!) since they would have required fairly complex contructs or, for instance, CORBA data types and functions (e.g. dealing with endianism). This means you have to prepare yourself a few lines of code. Details are provided later on in this manual.

The second level is the hardware. A new addition on an existing platform or a new platform altogether might require preparing a few YARP device drivers. These are to all effects classes that support the methods for accessing the hardware which is normally implemented through function calls to whatever provided by the hardware vendor. This comes typically in the form of either a DLL or a static library.

Finally, you might need to prepare the configuration files for a new robotic platform and unfortunately this requires fiddling with the {\bf libYARP\_robot} library.

You should not be scared by tweaking into YARP's code since the level of customization required is fairly simple and hopefully well documented. Once the appropriate classes are in place the compilation and link to application and tools should proceed smoothly -- or at least this has been our experience so far. In fact, YARP has been compiled and presently runs on four different operating systems -- meaning that it can run on a heterogeneous cluster of PCs -- and it has been customized to run on something like six different robots.

\subsection{Some numbers}
The following table contains some figures about the current usage of YARP for robot control.
\marginpar{We intended to give here a feeling of the type of computational architecture we're using.}
\begin{table}[h]
	\centering
		\begin{tabular}{|c|c|c|}
		\hline
			Robot name & Number of machines & OS \\
			\hline \hline
			Babybot (LIRA-Lab, Univ. of Genoa) & 13 & Windows, Qnx6 \\
			\hline
			Eurobot (LIRA-lab, Univ. of Genoa) & 11 & Windows, Qnx6 \\
			\hline
			Jenny (LIRA-Lab, Univ. of Genoa) & 3 & Windows \\
			\hline
			Obrero (MIT CSAIL) & 4 & Linux \\
			\hline
			Cog (MIT CSAIL) & 30 & Qnx4.25 \\
			\hline
		\end{tabular}
\end{table}

Although it is not completely evident from these numbers, YARP supports communication between dozens of processes distributed across a network of PCs. Communication proceeds through various protocols and shared memory when and where available. When specific communication protocols are available YARP uses them -- like Qnet under QNX6. YARP is not yet completely user friendly and it still requires you to track which processes each process is talking to. This, we believe, is fair enough when realtime control is an issue and you need to rely on data getting to the right place at all time. Moreover, many times you're allocating processes in order to get guaranteed performance. Indeed you might want to run your latest visual processing onto your fastest PC, and you are able to just do so.

A final note: YARP is not designed for message passing but rather for streaming communication. Unless special precautions are taken messages are not guaranteed to get through. Please bear this in mind to make sensible design choices early during your project.

\subsection{Licensing}
\marginpar{YARP might be migrating to GPL sometimes in the future}
YARP is distributed under the Academic Free License. The template of the academic
free licence can be found in \$YARP\_ROOT/conf/licence.template. If you're adding to YARP then you can either add the template to each file or simply make a link to it. There's also a shorter version called licence.short.template. The following is an excerpt from the license template file:

\begin{verbatim}
///
///
///       YARP - Yet Another Robotic Platform (c) 2001-2004 
///
///                    #Add our name(s) here#
///
///     "Licensed under the Academic Free License Version 1.0"
///
\end{verbatim}

You can of course customize this text by adding your name. The shorter version of the license template is more practical and thus preferred.