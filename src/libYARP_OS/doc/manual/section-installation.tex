
\section{Installation}

\label{sect:install}

We suggest that, if you are a first-time user of YARP, you install it
on one or two machines first, before trying to set it up for a cluster.

To set YARP up on a single machine, first install the ACE library,
then libYARP\_OS, then test your installation before moving on to 
other YARP components.  To set YARP up on a cluster, you may 
wish to place ACE and the YARP libraries in a common shared directory
rather than installing on each machine.  We have a few tips for this 
process, particularly if you are using multiple operating systems (as
is common because of hardware support constraints).

\subsection{Installing the ACE library}

We require ACE version 5.3.3 or greater.

On Debian-based linux machines, ACE is available as a binary package;
simply type ``apt-get install libace-dev'' as super-user.  Please
check (e.g. by searching at {\tt http://packages.debian.org}) that the version available is 5.3.3 or greater.  At the time of
writing, this is true for Debian unstable but not true for Debian
stable.

For all other scenarios, we suggest you go to the ACE homepage, and
download and compile the source package:

\begin{quote}
{\tt http://www.cs.wustl.edu/~schmidt/ACE.html}
\end{quote}

%On SuSE Linux, before compiling the ACE source, get packages ``openssl'' and
%``openssl-devel'' (from command-line: ``yast -i openssl'' and ``yast
%-i openssl-devel'') if they are not already present.

Since ACE is open-source, we have taken the liberty of redistributing 
a version of it that is known to work with YARP, available from our 
website:

\begin{quote}
{\tt http://yarp0.sourceforge.net}
\end{quote}




\subsection{Installing the libYARP\_OS library}


Download the latest package from the YARP sourceforge site:

\begin{quote}
{\tt http://yarp0.sourceforge.net}
\end{quote}

Unpack it and, err, install it.  Details to come once we 
actually make the release...

