
\section{Installation}

\label{sect:install}

We suggest that, if you are a first-time user of YARP, you install it
on one or two machines first, before trying to set it up for a cluster.

To set YARP up on a single machine, first install the ACE library,
then libYARP\_OS, then test your installation before moving on to 
other YARP components.  To set YARP up on a cluster, you may 
wish to place ACE and the YARP libraries in a common shared directory
rather than installing on each machine.  We have a few tips for this 
process, particularly if you are using multiple operating systems (as
is common because of hardware support constraints).

There are two fairly different ways of installing YARP. Again, if you are a first-time user then we strongly recommend to follow the procedure described in section \ref{sect:easy} next. Otherwise, we have also documented the full-blown installation procedure.

\section{Quick installation guide}
\label{sect:easy}

This section explains how to get to quickly run YARP and test the functionality of the libraries. The aim here is not to give a full account of what you can do but rather to shorten the time you spend on first-time installation. We reasoned that this might be enough for many users. The next step consists of getting all the required packages.

\subsection{Prerequisites to installation}

First of all, configure a CVS client as explained in SourceForge (http://www.sf.net).
You need to choose a directory (that will become the \$YARP\_ROOT directory) where to download the YARP source code. Make sure you check the ``prune'' flag in CVS so you don't create empty directories. In Windows you might want to use WinCVS.

We recommend WinCVS 1.3.XX that solves the annoying timezone-daylight saving time
problem (earlier versions had an issue twice a year!). We used also {\em putty} (and {\em plink}) as a ssh client for WinCVS as suggested on SourceForge. We found this configuration very stable and we suggest replicating it if you're not an expert. 

If you decide to use {\em putty} then make sure you download all the small applications that come with the main terminal application. You will need a few of them during the configuration procedure (e.g. {\em putty}, {\em pagent}, and {\em plink}). Please make sure you do not change the default from the terminal application (ssh) because it applies to {\em plink} too.

The latest WinCVS version has its own ssh implementation internally so that you don't need the external ssh client. Please, also configure and upload the public-private key as requested by SourceForge (unless you're an anonymous user). Make sure your CVS ssh connection is functional before moving to the next section. 

Since we're checking prerequisites here, download ACE 5.4.2 from:

\begin{quote}
{\tt http://www.cs.wustl.edu/\symbol{126}schmidt/ACE.html}
\end{quote}

or from our download page at:
\begin{quote}
{\tt http://yarp0.sourceforge.net/software.html}
\end{quote}

It comes as a single {\em tar.gz} file that you can open both on Linux and Windows. Keep it handy since we're going to use it again.

\subsection{Installing ACE}
We require ACE version 5.3.3 or greater (but we recommend 5.4.2). ACE should compile fine just by following the installation instructions that can be found on the website (http://www.cs.wustl.edu/\symbol{126}schmidt/ACE.html). 

On Debian-based linux machines, ACE is available as a binary package; simply type ``apt-get install libace-dev'' as super-user.  Please check (e.g. by searching at {\tt http:// packages.debian.org}) ~that the version available is 5.3.3 or greater.  At the time of writing, this is true for Debian unstable but not true for Debian stable.

If you are not on a Debian linux and have downloaded {\em ACE-5.4.2.tar.gz}, copy it to \$YARP\_ROOT and then follow the steps indicated next to compile and install ACE manually:
\marginpar{There's plenty of documentation of this procedure on the ACE website.}
\begin{itemize}
\item gunzip the file \ctt{gunzip ACE-5.4.2.tar.gz};
\item untar the file \ctt{tar xvf --directory src/}; this will put the extracted files under \$YARP\_ROOT/src/ACE\_wrappers;
\item \ctt{cd src/ACE\_wrappers/include/makeinclude}, type \ctt{ln -s platform\_linux.GNU platform\_macros.GNU} and then press return;
\item \ctt{cd ../../ace}, type \ctt{ln -s config-linux.h config.h};
\item type \ctt{make static\_libs\_only=1 debug=0} and return;
\end{itemize}

You should see something like this:
\begin{verbatim}

GNUmakefile: /usr/src/yarp/src/ACE_wrappers/ace/GNUmakefile.ACE 
MAKEFLAGS=w -- static_libs_only=1 debug=0

g++ -W -Wall -Wpointer-arith -O3   -pipe    -D_REENTRANT 
-DACE_HAS_AIO_CALLS -D_GNU_SOURCE   -I/usr/src/yarp/src/ACE_wrappers 
-DACE_HAS_EXCEPTIONS -D__ACE_INLINE__ -I.. -DACE_HAS_ACE_TOKEN 
-DACE_HAS_ACE_SVCCONF -DACE_AS_STATIC_LIBS  -c  -o .obj/Local_Name_Space.o
Local_Name_Space.cpp
g++ -W -Wall -Wpointer-arith -O3   -pipe    -D_REENTRANT 
-DACE_HAS_AIO_CALLS -D_GNU_SOURCE   -I/usr/src/yarp/src/ACE_wrappers 
-DACE_HAS_EXCEPTIONS -D__ACE_INLINE__ -I.. -DACE_HAS_ACE_TOKEN 
-DACE_HAS_ACE_SVCCONF -DACE_AS_STATIC_LIBS  -c  -o .obj/Name_Proxy.o 
Name_Proxy.cpp

\end{verbatim}

and so forth. When done, install everything by doing:

\begin{itemize}
\item \ctt{cp *.cpp \$YARP\_ROOT/include/ace};
\item \ctt{cp *.h \$YARP\_ROOT/include/ace};
\item \ctt{cp *.inl \$YARP\_ROOT/include/ace};
\item \ctt{cp -R os\_include/ \$YARP\_ROOT/include/ace};
\item \ctt{cp *.a \$YARP\_ROOT/lib/linux}.
\end{itemize}

As for any library you must make sure the header files are in the compiler path. By default YARP makefiles assume ACE files are in \$YARP\_ROOT/include/ace. Make sure you copy all of them there (this is what is written in the short list above). Also the ACE library is copied to an appropriate place. 

On Windows you can unzip the ACE distribution and simply open the {\em .dsw} workspace file from within Visual Studio and then start the compilation. Change directory to:

\begin{quote}
{\tt \%YARP\_ROOT\%$\backslash$src$\backslash$ACE\_wrappers$\backslash$ace}
\end{quote}
 
\noindent and open {\em ACE.dsw}. Choose your configuration (e.g. ACE -- Win32 Debug) and then build it. Proceed with the same manual copy of source files into \%YARP\_ROOT\%$\backslash$include$\backslash$ace as before. Also look for the {\em *.lib} files and copy them into \%YARP\_ROOT\%$\backslash$lib$\backslash$winnt. Close the workspace when done.

\subsection{Environment variables}
ACE requires defining the \$ACE\_ROOT environment variable on Linux. Also, it is convenient to define a root directory for YARP. On a typical Linux box, you can add the following definitions to your {\em .profile} shell login file:
\marginpar{While this is not strictly necessary, we recommend doing it for once.}

\begin{verbatim}
export YARP_ROOT=/usr/src/yarp
export CVSROOT=:ext:youruser@cvs.sourceforge.net:/cvsroot/yarp0
export CVS_RSH=/usr/bin/ssh
export ACE_ROOT=$YARP_ROOT/src/ACE_wrappers
export LD_LIBRARY_PATH=/usr/local/lib
export PATH=$PATH:$YARP_ROOT/bin/linux:$HOME/bin
export PATH=/usr/local/bin:$PATH
export PATH=.:/lib:$PATH
\end{verbatim}

Otherwise just type the same {\em export} commands on a terminal window. Note also as ACE resides in a subdirectory of the \$YARP\_ROOT. On Windows you go to the control panel, click on the {\em system} icon and search for the {\em environment variables} button on one of the tabs. By clicking this button opens a new window where you can add environment variables. There are two options here; we recomment adding them to the system wide variables rather than to the current interactive user only. 

\subsection{Compiling libYARP\_OS}

On Linux simply go to \$YARP\_ROOT and type: \ctt{make}. The makefile contains also commands for completing the installation (see the printout below). 
\begin{verbatim}

a - ../obj/linux/YARPThreadAux.o
a - ../obj/linux/YARPThread.o
a - ../obj/linux/YARPTime.o
mkdir -p /usr/src/yarp/lib/linux
mkdir -p /usr/src/yarp/include/yarp
cp ../lib/linux/libYARP_OS.a /usr/src/yarp/lib/linux
cp -f ../include/yarp/*.h /usr/src/yarp/include/yarp
make[1]: Leaving directory `/usr/src/yarp/src/libYARP_OS/src'

\end{verbatim}

If you reached so far it means you're done with the library. Before testing YARP you need a few utilities that are in \$YARP\_ROOT/src/tools and need to be compiled of course. Change directory to \$YARP\_ROOT/src/tools and do:
\begin{itemize}
\item \ctt{cd yarp-connect/; make; cd ..};
\item \ctt{cd yarp-names/; make; cd ..};
\item \ctt{cd yarp-read/; make; cd ..};
\item \ctt{cd yarp-service/; make; cd ..};
\item \ctt{cd yarp-write/; make; cd ..}.
\end{itemize}

The last piece of output should look like:
\begin{verbatim}

mkdir -p obj/linux
g++ -O3  -I/usr/src/yarp/include/linux -I/usr/src/yarp/include -I.  
-fexceptions -pipe -Wpointer-arith -Wno-uninitialized  -D__LINUX__ 
-DYARP_OS_CONFIG=LINUX -I/usr/include/g++/ -I/usr/src/yarp/include/ 
-DACE_NDEBUG -DACE_USE_RCSID=0 -DACE_HAS_EXCEPTIONS -D__ACE_INLINE__ 
-DACE_HAS_ACE_TOKEN -DACE_HAS_ACE_SVCCONF -DACE_AS_STATIC_LIBS   
-c yarp-write.cpp -o obj/linux/yarp-write.o
mkdir -p ../bin/linux
g++ -O3  -I/usr/src/yarp/include/linux -I/usr/src/yarp/include -I.  
-fexceptions -pipe -Wpointer-arith -Wno-uninitialized  -D__LINUX__ 
-DYARP_OS_CONFIG=LINUX -I/usr/include/g++/ -I/usr/src/yarp/include/ 
-DACE_NDEBUG -DACE_USE_RCSID=0 -DACE_HAS_EXCEPTIONS -D__ACE_INLINE__ 
-DACE_HAS_ACE_TOKEN -DACE_HAS_ACE_SVCCONF -DACE_AS_STATIC_LIBS   
obj/linux/yarp-write.o -o ../bin/linux/yarp-write 
-L/usr/src/yarp/lib/linux  -lYARP_OS -lACE -lpthread -ldl
mkdir -p /usr/src/yarp/bin/linux
cp ../bin/linux/yarp-write /usr/src/yarp/bin/linux

\end{verbatim}

The executables are installed in \$YARP\_ROOT/bin/linux and since it is in the path, you can now run them. The next step is to go to chapter \ref{chapter:usage} and test your installation. 


On Windows open the {\bf libYARP\_OS} workspace located in:

\begin{quote}
{\tt \%YARP\_ROOT\%$\backslash$src$\backslash$libYARP\_OS$\backslash$src}
\end{quote}

\noindent and choose the configuration as usual (e.g. Debug or Release). Start the build process. When finished you have to manually install files as before:

\begin{itemize}
\item copy all files in ..$\backslash$include$\backslash$yarp into \%YARP\_ROOT\%$\backslash$include$\backslash$yarp
\item copy all libraries (*.lib) in ..$\backslash$lib$\backslash$winnt into
\%YARP\_ROOT\%$\backslash$lib$\backslash$winnt
\end{itemize}

Now change directory to {\em tools} and open the {\em tools-os.dsw} workspace. Compile all the tools in the workspace; then look into:

\begin{quote}
{\tt .$\backslash$obj$\backslash$*.exe}
\end{quote}

\noindent and copy all of them into:

\begin{quote}
{\tt \%YARP\_ROOT\%$\backslash$bin$\backslash$winnt}
\end{quote}

If everything compiled and no errors showed up then the next step is to go to chapter \ref{chapter:usage} and test your installation. 


\section{Installing YARP: general method}
Sometimes the level of configuration you would like to consider requires the understanding of the more difficult way of installing YARP. This might be scary as a first approach to a new tool especially since there is an infinite number of things that can fail which are difficult to debug if you're not aware of the details of the libraries. Definitely, this is too much for making a tool a good one. On the other hand, sometimes, complex hardware requires complex control. The next sections document a general procedure to install YARP by using a set of {\em perl} scripts provided with the distribution.

\subsection{Why doesn't YARP follow installation standards?}
You might be wondering why we haven't set things up as in many existing open source packages. A short answer is that dealing with multiple operating systems is messy and we haven't found any neat and clean way of doing it. The best we came about is using perl for preparing the installation. See section \ref{sect:perl}.

\subsection{Perl scripts}
\label{sect:perl}

YARP uses perl scripts for configuration and installation. We chose to do so to make compilation on Windows and Unixes as uniform as possible in spite of the differnce between makefiles and Microsoft Developer Studio project files. Perl is generally available on Linux and available for Windows in a very nice free distribution. Check the following website:

\begin{quote}
{\tt http://www.activestate.com/Products/ActivePerl/}
\end{quote}

At the moment of writing we use version 5.8.4.810. Once intalled, please make sure the environment variables are properly registered so that you can run perl from the command line.
Type \ctt{perl} on the command line to check whether it works.

\subsection{Prerequisites to installation}

First of all, configure a CVS client as explained in SourceForge (http://www.sf.net).
You need to choose a directory (the prospective \$YARP\_ROOT) where to download the
YARP source code. Make sure you check the ``prune'' flag in CVS so you don't create
empty directories. In Windows you might want to use WinCVS.

We recommend WinCVS 1.3.something that solves the annoying timezone-daylight saving time
problem (earlier versions had an issue twice a year!). We used also {\em putty} (and {\em plink}) as a ssh client for WinCVS as suggested on SourceForge. We found this configuration very stable and we strongly suggest replicating it if you're not an expert. 
If you decide to use putty then make sure you download all the small applications that come with the main terminal application. You will need a few of them during the configuration procedure (e.g. {\em putty}, {\em pagent}, and {\em plink}). The latest WinCVS version has its own ssh implementation internally so that you don't need the external ssh client.

If you use {\em putty}, please make sure you do not change the default from the putty terminal application (ssh)
because it applies to plink too. Please, also configure and upload the public-private key as requested by SourceForge (unless you're an anonymous user). Make sure your CVS ssh connection is functional before moving to the next section. Since we're checking prerequisites here, download ACE 5.4.2 from:

\begin{quote}
{\tt http://www.cs.wustl.edu/\symbol{126}schmidt/ACE.html}
\end{quote}

or from our download page at:
\begin{quote}
{\tt http://yarp0.sourceforge.net/software.html}
\end{quote}

It comes as a single {\em tar.gz} file that you can open both on Linux and Windows. Keep it handy since we're going to use it again. YARP includes now header files (from cvlibrary) and DLL for the Intel IPL (optionally used in image processing) in the repository so you no longer need to download and install them separately.

DLLs from Visual Studio 6.0: strangely enough the default install doesn't have the
correct version (e.g. the C terminal library wouldn't normally work. This applies to
NT4.0; I haven't investigated the issue on 2K or XP. If you're running NT4.0 you might
need to install Visual Studio on each machine of the network to get the DLLs right.

Please spend some time reading the SourceForge documentation on the use of CVS and
the WinCVS documentation. You should become acquainted with the different checkout
options and with the WinCVS configuration. It will definitely help later during
development.

Directories cannot be removed from the CVS server. Make sure you know what
you're doing when adding a new directory to the repository (admitting you're allowed to).
For Windows users, your machine should have the Visual Studio 6.0 installed and properly
service packed.

A few environment variables are required to properly compile YARP. The perl scripts will check whether \$YARP\_ROOT is defined and will complain otherwise. Also, many applications rely on having a \$YARP\_ROOT defined to locate configuration files. You're requested to define:
\begin{itemize}
\item \$YARP\_ROOT (\%YARP\_ROOT\% in Windows) environment variable (e.g. \ctt{export YARP\_ROOT=/usr/src/yarp}).
\item and then add to path \$YARP\_ROOT/bin/(your OS)/ (location of binaries and DLLs), and of course substitute the forward slash with the backslash where appropriate.
\item add also to the path: \$YARP\_ROOT/scripts (and the subdirectories you need). In Windows I tend to prefer adding them to the ``system'' environment and not to the current user only.
\end{itemize}

On Windows you have to check that you can run the developer studio from the command line. The usual check is by typing \ctt{msdev} on the console. If it fails then you might need to go and look for the {\em VCVARS32.bat} and run it from your terminal. The {\em VCVARS32.bat} script prepares the environment to run visual studio and to compile from the command line.

On Unixes (or Qnx) you can add the environment initialization in the {\em .profile} in your home directory. An example of a plausible environment initialization is shown below. As you can see ACE on Unixes requires the definition of the \$ACE\_ROOT environment variable.

\begin{verbatim}
export YARP_ROOT=/usr/src/yarp
export ACE_ROOT=$YARP_ROOT/src/ACE_wrappers
export CVSROOT=:ext:youruser@cvs.sourceforge.net:/cvsroot/yarp0
export CVS_RSH=/usr/bin/ssh
export LD_LIBRARY_PTH=/usr/local/lib
export PATH=$PATH:$YARP_ROOT/bin/qnx6:$HOME/bin
export PATH=/usr/local/bin/:$PATH
export PATH=.:/lib:$PATH
\end{verbatim}

A brief checklist on this procedure is provided here for convenience:
\begin{enumerate}
	\item Download \& install the CVS client and configure your sourceforge user.
	\item Checkout YARP \ctt{cvs checkout -P yarp} somewhere on your directory tree (this will be your \$YARP\_ROOT.
	\item Install perl if your operating system requires to do so (e.g. Windows).
	\item Download ACE, possibly version 5.4.2. I tend to like the tar.gz file that has Unix-like newline characters.
	\item Setup environment variables as explained above.
\end{enumerate}

\subsection{Getting acquainted with the YARP directory tree}
The YARP directory tree contains the usual src, lib, bin, etc. subdivision. There are a few differences from a standard distribution bacause of the handling of many libraries and the support of several operating systems. The following table represents a quick tour into the directory hierarchy:

\begin{table}[h]
	\centering
		\begin{tabular}{|c|c|}
		\hline
			Directory name & Description \\
			\hline \hline
			bin & binaries with subdirectories for multiple OS \\
			\hline
			conf & configuration files, templates, etc. \\
			\hline
			docs & installation manuals, tutorial and other documents \\
			\hline
			include & header files are moved here after compilation of libraries \\
			\hline
			lib & libraries are moved here after compilation \\
			\hline
			packaging & scripts for creating the YARP distributions (does not concern the user) \\
			\hline
			scripts & scripts of various robotic experiments (not too neat to have them here) \\
			\hline
			src & the main source subdirectories including libraries, tools and experimental code \\
			\hline
		\end{tabular}
\end{table}


\subsection{Installing the ACE library}
\label{sect:ace}

We require ACE version 5.3.3 or greater (but we recommend 5.4.2). There are two alternatives for compiling ACE: i) follow the ACE installation instructions, and ii) allow the YARP perl scripts to compile ACE for you -- the latter works only with ACE 5.4.2.

If you are taking the first route, on Debian-based linux machines, ACE is available as a binary package; simply type ``apt-get install libace-dev'' as super-user.  Please
check (e.g. by searching at {\tt http:// packages.debian.org}) ~that the version available is 5.3.3 or greater.  At the time of writing, this is true for Debian unstable but not true for Debian stable.

For all other scenarios, we suggest you go to the ACE homepage, and
download and compile the source package:

\begin{quote}
{\tt http://www.cs.wustl.edu/\symbol{126}schmidt/ACE.html}
\end{quote}

%On SuSE Linux, before compiling the ACE source, get packages ``openssl'' and
%``openssl-devel'' (from command-line: ``yast -i openssl'' and ``yast
%-i openssl-devel'') if they are not already present.

Since ACE is open-source, we have taken the liberty of redistributing 
a version of it that is known to work with YARP, available from our 
website:

\begin{quote}
{\tt http://yarp0.sourceforge.net/software.html}
\end{quote}

This is the {\em companion software} page; it contains links to various packages we found useful when working with YARP. And, surprise surprise, you can download version 5.4.2 from this page.

If you trust us, and you'd like to go through the building of ACE, you can unzip it into a directory within YARP (we recommend \$YARP\_ROOT/src/ACE\_wrappers). The unzip complains a bit since certain examples are included twice into the tar.gz file (ACE's fault!).

You're done for now; ACE will be compiled and installed when compiling the {\bf libYARP\_OS} library. Note that the ACE\_wrapper directory is already in the repository and contains a perl script called {\em build.pl} and a readme file {\em readme.txt}. Now it's time to proceed to the next step.

\subsection{Installing the libYARP\_OS library}
Compilation is best carried out by running the perl scripts. Each library contains two scripts, one for doing the configuration and another for the actual build. Depending on the operating system the scripts call either {\em make} or {\em msdev}. Each library directory contains both the makefile and the msdev project files.

By invoking the configuration script {\em configure.pl} a certain number of parameters influencing compilation is stored in \$YARP\_ROOT/conf/context.conf. An example of a few lines extracted from the context file is shown below:

\begin{verbatim}
[Compile_OS]
ACE_PATH= $YARP_ROOT/src/ACE_wrappers
ACE_Rebuild= NO
Lib_Clean= TRUE
Lib_Debug= TRUE
Lib_Install= TRUE
Lib_Release= TRUE
Tools_Debug= FALSE
Tools_Rebuild= YES
\end{verbatim}

These parameters are read by the {\em build.pl} script to compile libraries accordingly. In particular to compile the {\bf libYARP\_OS} you should type at the terminal:

\begin{verbatim}
cd $YARP_ROOT/src/libYARP_OS
perl configure.pl
\end{verbatim}

the second line runs the perl script. If the script is made executable you can save five strokes and run it without the leading {\em perl}. As a consequence, the configuration procedure is started; you should see:

\begin{verbatim}
Entering configure process of YARP environment...
'uname' is not recognized as an internal or external command,
operable program or batch file.
Ready to start...
Now I'm going to ask a few questions to help the configuration. So, let's start.
..
For pathnames you can use (type) the pre-defined value $YARP_ROOT that I've veri
fied as: "D:\Users\pasa\Repository\yarp"

Please, use always the forward slash as separator!
I determined already that you're running on a supported OS: winnt
What's your OS? [winnt]
\end{verbatim}

recognized identifiers for the operating system tag are:
\begin{itemize}
\item winnt, for Win32 compilation
\item linux, for Linux
\item qnx6, for Qnx6
\item darwin, for MacOs
\end{itemize}

by hitting {\em return} you simply accept the default/proposed entry which is indicated in square brackets (which is {\em winnt} in this example). The following example shows what required and asked next:

\begin{verbatim}
Where has ACE been unpacked? [$YARP_ROOT/src/ACE_wrappers]
Do you want to rebuild ACE, i.e. clean before building? [NO] YES
Would you like to set a default for library compilation?
Clean first: i.e. rebuild libraries? [TRUE]
Debug mode? [TRUE]
Release mode (optimization on)? [TRUE]
Install after compile? [TRUE]
Would you like to rebuild the YARP tools [YES]
Would you like to compile the tools for debugging? [FALSE]
We're done for now, the context file is being created: 
"D:\Users\pasa\Repository\yarp/conf/context.conf"
Done!
\end{verbatim}

In particular, the ACE root directory is requested. The {\bf libYARP\_OS} then locates an additional helper script within the ACE root and possibly runs it to compile ACE (as mentioned in section \ref{sect:ace}). It is then asked whether to actually rebuild ACE (by doing a {\em make clean} for instance). Valid replies to yes/no questions are strings of the type of YES, yes, YE, Y, etc. and similarly TRUE, T, t for true/false.

It is asked whether to rebuild the library in question {\bf libYARP\_OS} and whether to enable the debug and optimization flags. In Windows both libraries, the debug and the optimized version, can be created and they will happily cohexist with different names in the same directories. The {\bf libYARP\_OS} contains also a few tools required for the system to function. The script asks whether to compile also these tools and to compile them for debugging.

It finally prints the filename of the context file which is now up to date. You can now type:

\begin{verbatim}
perl build.pl
\end{verbatim}

and press return. The build script runs; its output should look similar to this one (this is a Windows example):

\begin{verbatim}
Entering compile process of YARP OS libraries...
'uname' is not recognized as an internal or external command,
operable program or batch file.
Ready to start...
Working with: D:\Users\pasa\Repository\yarp/conf/context.conf
Looking for ACE...
Entering build process for ACE libraries...
'uname' is not recognized as an internal or external command,
operable program or batch file.
I'm trying to guess the project file name
Using ACE.dsp for project ACE


Cleaning...
Deleting intermediate files and output files for project 'ACE - Win32 Debug'.
Deleting intermediate files and output files for project 'ACE - Win32 Release'.


Compiling debug
--------------------Configuration: ACE - Win32 Debug--------------------
Compiling resources...
Compiling...
Local_Name_Space.cpp
Name_Proxy.cpp
Name_Request_Reply.cpp
Name_Space.cpp
Naming_Context.cpp
NT_Service.cpp

\end{verbatim} 

which means that compilation of ACE is started. It'll take a while depending on your machine. The build script changed to the ACE directory and ran the build script there. When it's done you should see something like this:

\begin{verbatim}
Timer_Wheel.cpp
Metrics_Cache.cpp
Generating Code...
Linking...
   Creating library ..\lib/ACEd.lib and object ..\lib/ACEd.exp

ACEd.dll - 0 error(s), 0 warning(s)

Compiling release
--------------------Configuration: ACE - Win32 Release--------------------
Compiling resources...
Compiling...
Local_Name_Space.cpp

\end{verbatim}

which means that the debug version of ACE has been compiled succesfully and compilation of optimized code started. When also ACE with optimization flags is compiled, the script copies the ACE header files into a common store where every YARP header will be placed and the libraries under \$YARP\_ROOT/lib/winnt or any other allowed operating system name. This doesn't mean packages will share a single common directory. ACE headers will be included by:

\begin{verbatim}
#include <ace/Acceptor.h>
\end{verbatim}

provided the common store directory is specified to the compiler: \$YARP\_ROOT/include/. This means ace headers are all under \$YARP\_ROOT/include/ace. YARP headers are included, for example, as:

\begin{verbatim}
#include <yarp/YARPThread.h>
\end{verbatim}

Finally, after a succeful compilation of ACE, the script will change back to the {\bf libYARP\_OS} directory and start the compilation of YARP proper. It generates the libYARP\_OS.lib and copies header files under \$YARP\_ROOT/include/yarp and libraries and binaries under \$YARP\_ROOT/lib/winnt and \$YARP\_ROOT/bin/winnt respectively. Operating system names and subdirectory names coincide. The allowed names are: {\em winnt}, {\em linux}, {\em qnx6}, and {\em darwin}. The very last steps are shown in the example below:

\begin{verbatim}
--------------------Configuration: yarp_names - Win32 Release-------------------
-
Compiling...
yarp-names.cpp
Linking...
LINK : warning LNK4089: all references to "WINMM.dll" discarded by /OPT:REF

yarp-names.exe - 0 error(s), 1 warning(s)

Installing YARP_OS tools to default install directory.
Copying ./yarp-connect/obj/winnt/yarp-connect.exe
Copying ./yarp-read/obj/winnt/yarp-read.exe
Copying ./yarp-service/obj/winnt/yarp-service.exe
Copying ./yarp-write/obj/winnt/yarp-write.exe
Copying ./yarp-names/obj/winnt/yarp-names.exe

\end{verbatim}

This same procedure is generic and will work fine for all supported operating systems. The best thing to do at this point is to test YARP on a single machine to make sure everything is working fine. The next chapter is devoted to moving the first few steps into the world of YARP.


\section{Caveats}
\label{sect:caveats}

Before testing the YARP installation you might want (it is not mandatory but recommended) to configure the YARP name server. This is an application that allocates symbolic names to network ports and addresses. YARP code talks to the name server for determining where and how to connect two specific ports together. The following text is an example of a name server configuration file. A template is provided in \$YARP\_ROOT/conf/ and it is called {\em namer.conf.template}. The actual configuration file must reside in \$YARP\_ROOT/conf/ and named {\em namer.conf}.

\begin{verbatim}
130.251.43.254 10000

// start network description, don't forget to separate "Node=" 
// and names with a space

[NETWORK_DESCRIPTION]
Node= leto    local 127.0.0.1 127.0.0.1	
              default 130.251.43.134 130.251.43.134	
              Net0 130.251.43.134 130.251.43.134	
              Net1 10.0.1.134 10.0.1.134
Node= euterpe	local 127.0.0.1 127.0.0.1	
              default 130.251.43.135 130.251.43.135	
              Net0 130.251.43.135 130.251.43.135	
              Net1 10.0.1.135 10.0.1.135
Node= oceanus	local 127.0.0.1 127.0.0.1	
              default 130.251.43.136 130.251.43.136	
              Net0 130.251.43.136 130.251.43.136	
              Net1 10.0.1.136 10.0.1.136
Node= thaumas	local 127.0.0.1 127.0.0.1	
              default 130.251.43.129 130.251.43.129	
              Net0 130.251.43.129 130.251.43.129
Node= hades   local lo0 127.0.0.1		
              default en0 130.251.43.130			
              Net0 en0 130.251.43.130			
              Net1 en1 10.0.1.130
[END]
\end{verbatim}

The first line describes the name server configuration. The first number is the IP address of the machine where the server is running. Note that this is not your DNS! It's the YARP name server address. The second number is the port the name server responds to ($10000$ in this example).

The file contains then a series of machine definitions. For each ``node'' you should define a machine name (not actually used for anything sensible) and a list of networks, interface names, and IP addresses. In the first entry, for example, a machine called {\em leto} is defined, it has a {\em local} network being the loopback, a {\em default} one in the subnetwork $130.251.4.$ which is an alias of {\em Net0} and finally a private {\em Net1} with address $10.0.1.$. For each network entry there are two numbers. The first represents the interface name which on Windows is equivalent to the IP address, the second is the IP address proper.
This last point is perhaps clear if you look at the last node entry. This is a machine called {\em hades}. You can note here that each network entry contains the inteface name of the Ethernet card the network is connected to. On Linux type \ctt{ifconfig} to get the interface names.

