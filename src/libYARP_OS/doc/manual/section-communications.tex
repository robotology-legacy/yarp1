

%YARP's communication code is, at heart, simply about moving bytes
%along pipes between processes, where the individual pipes and
%processes can be created and destroyed with minimal impact on the
%rest of the system.  For other communication models, YARP is not
%the tool to use -- try ACE/TAO for example.

\section{Getting started}

Make sure you have installed libYARP\_OS as described in
Section~\ref{sect:install}.  We will now exercise your installation to
make sure everything is okay and correctly configured.  First, please
open two terminal windows; we'll refer to these as A and B.
%
Before doing any communication, is is necessary to start the YARP name
service -- this is a program which keeps track of what YARP resources
are currently available and how to access them.
%
In terminal A, type:

\begin{verbatim}
yarp-service --run
\end{verbatim}

If all goes well, you will see a series of messages finishing in ``***
YARP name service started OK'' -- if not, check the configuration file
mentioned in the messages, or follow the advice given.
%
Next, in terminal B, type:
%
\begin{verbatim}
yarp-service --check
\end{verbatim}
%
This will try to communicate with the name server, send a message,
receive a message, and basically make sure that everything is
working well.  Here is what you should see:

\begin{quote}
{\small
\begin{verbatim}
=================================================================
=== Trying to register some ports
*** making port /yarp-service/1 on default network [tcp, 11 ports]
*** making port /yarp-service/2 on default network [tcp, 11 ports]
=================================================================
=== Trying to connect some ports
*** connecting /yarp-service/1 to /yarp-service/2
=================================================================
=== Trying to write some data
*** Writing number 42
=================================================================
=== Trying to read some data
*** Read number 42
=================================================================
=== Trying to unregister some ports
=================================================================
*** YARP seems okay!
\end{verbatim}
}
\end{quote}
%
\noindent
Here is an example of things going wrong:
%
\begin{quote}
{\small
\begin{verbatim}
=================================================================
=== Trying to register some ports
*** making port /yarp-service/1 on default network [tcp, 11 ports]
*** FAILED port /yarp-service/1 - is the YARP name server running?
*** making port /yarp-service/2 on default network [tcp, 11 ports]
*** FAILED port /yarp-service/2 - is the YARP name server running?
=================================================================
=== Trying to connect some ports
==================================================================
=== Trying to write some data
*** Writing number 42
=================================================================
=== Trying to read some data
=== Trying to read some data
=== Trying to read some data
=================================================================
=== Trying to unregister some ports
=================================================================
*** YARP seems broken.
*** Check the configuration file at [/home/paulfitz/.yarp/conf/namer.conf]
*** It currently specifies a server at:
                Machine [wuffles]
                Port number [10000]
*** These settings can be changed on the first line of the file.
*** Perhaps try using an IP address instead of a host name.
***
*** If this is the wrong configuration file,
***    please set the optional YARP_ROOT environment variable
***    to the directory in which the desired namer.conf resides
***
\end{verbatim}
}
\end{quote}

If networking is broken on your machine, then YARP will have trouble.
If you are running the YARP name server on a machine that does not
have a static IP address or a name registered with DNS, then you
should check the configuration file mentioned by yarp-service, and
make sure that the first line of the configuration file is of the form:
\begin{quote}
  current.valid.ip.address 10000
\end{quote}

The number ``10000'' is the default port number that the YARP name
service uses.  This is usually fine; if you have a conflict with this,
then please change it in the configuration file.

If all is well, we can try exercising YARP a little more.

In terminal B, type:
%
\begin{verbatim}
yarp-read /port1
\end{verbatim}
%
This will report \ctt{making port /port1 ...} in terminal B and in
terminal A some corresponding message will also appear.
%
Next, in terminal C, type:

\begin{verbatim}
yarp-write /port2
\end{verbatim}

If all goes well, this will report \ctt{making port /port2} in
terminal C and in terminal A some corresponding message will also
appear.
%
Now, in terminal D, type:

\begin{verbatim}
yarp-connect /port2 /port1
\end{verbatim}

Terminal C will report \ctt{connecting /port2 to /port1}.  Now if
you type ``hello world'' into terminal C, and hit enter, that text
will appear on terminal B.

If you have another computer with YARP installed on it, open a
terminal on it (let's call that terminal E) and type:

\begin{verbatim}
yarp-service --remote machinename
\end{verbatim}

where machinename is the full name or (better still) IP address of the
first machine on which the YARP name service is running.  If all goes
well it should report \ctt{connection established with YARP name
service running on {\bf machinename}}.  If things do not go well, 
try running ``yarp-service --check'' in terminal E.  You now have
two configuration files, one on each machine, which should both
refer to the IP address of the first machine (since that is the one
on which the server is running on).

If everything is going okay so far, type on terminal E:
%
\begin{verbatim}
yarp_read /port3
\end{verbatim}
%
and on terminal D, type:
%
\begin{verbatim}
yarp_connect /port2 /port3
\end{verbatim}
%
Now any line you enter in terminal C should appear in both terminal B and E.



\subsection{Getting coding}

Suppose you have a structure \ctt{Target} in process A with some data
you would like to have in process B:
%
\begin{verbatim}
struct Target {
  int x, y, z;
  char label[20];
};
\end{verbatim}
%
If you are operating in a homogeneous network (same OS and hardware on
all machines), then moving Target structures around is easy.  See
section XXX for a few details needed otherwise.
%
The steps are create, register, communicate, unregister (optional).
The necessary methods and classes are in the following header file:

\begin{verbatim}
#include <yarp/YARPPort.h>
\end{verbatim}
%
Make sure that YARP is set up okay on the machines you are using by 
running \ctt{yarp-service --check}.  If not, go to section XXX.
%
In process A you would create an output port that can send a Target:
%
\begin{verbatim}
YARPOutputPortOf<Target> out_port;
\end{verbatim}
%
%
In process B you would create an input port that can receive a Target:
%
\begin{verbatim}
YARPInputPortOf<Target> in_port;
\end{verbatim}
%
Then, in the main() methods of each process, you would create a name
for each port so that they can find each other.  For example, in
process A:
%
\begin{verbatim}
out_port.Register("/target/out");
\end{verbatim}
%
and in process B:
%
\begin{verbatim}
in_port.Register("/target/in");
\end{verbatim}
%
Once both processes are running, these ports can be connected by
typing \ctt{yarp-connect /target/out /target/in} on the command line.
Alternatively, you can connect them using code; for example, from
process A call:
%
\begin{verbatim}
out_port.Connect("/target/in");
\end{verbatim}
%
Now the ports are connected and ready to communicate.  In process A
you could set up a Target as follows:
%
\begin{verbatim}
Target& target = out_port.Content();
target.x = target.y = target.z = 10;
strncpy(target.label,"hello yarp world",sizeof(target.label));
\end{verbatim}
%
And then you could send it like this:
%
\begin{verbatim}
out_port.Write();
\end{verbatim}
%
In process B, you can wait for the Target to arrive by calling Read():
%
\begin{verbatim}
bool success = in_port.Read();
\end{verbatim}
%
If a target is successfully read, it becomes available using Content():
%
\begin{verbatim}
if (success) {
  Target@ target = in_port.Content();
  cout << "target is located at (" << target.x
       << "," << target.y
       << "," << target.z
       << ") with label " << target.label << endl;

}
\end{verbatim}

%See Appendix XXX and demos in directory YYY for this code.


%\subsection{The details}

%Quality of service. 

%Crossing OS/Hardware/Compiler boundaries.

%Different protocols -- shared mem, tcp, udp, multicast.



\subsection{Messages in bottles}

When completely maximal efficiently is not a concern, it can be 
convenient to not have to compile the type of every message.
For this, the \ctt{YARPBottle} class is available.  This is a generic
container into which you can throw integers, floating point numbers,
and strings, send them across the network, then easy parse the message
again at the far side.  For example, the following code:
%
\begin{verbatim}
#include <stdio.h>

#include <yarp/YARPPort.h>
#include <yarp/YARPBottle.h>

int main() {
  YARPOutputPortOf<YARPBottle> out_port;
  out_port.Register("/test/demo00");
  out_port.Connect("/test/reader");

  for (int i=0; i<5; i++) {
    char buf[256];
    sprintf(buf,"test number %d", i);
    out_port.Content().writeText(buf);
    out_port.Write(1);
  }
  out_port.FinishSend();

  return 0;
}
\end{verbatim}
%
will send a series of YARPBottles out on the port \ctt{/test/demo00}.
If, before starting this code, you start a reader in another terminal with:
%
\begin{verbatim}
yarp-read /test/reader
\end{verbatim}
%
then the following messages should appear:
%
test number 0
test number 1
test number 2
test number 3
test number 4
%
The \ctt{yarp-read} reader responds to YARPBottles that contain text.
You can look at the source of yarp-read as an example of how to read 
a YARPBottle.

%explain: Write(1), FinishSend

\section{Further documentation}

At the moment of writing more documentation is available on the YARP website:

\begin{quote}
{\tt http://yarp0.sf.net/doc-yarp0/}
\end{quote}

You can find there the latest PDF file of this document, an html version of it and the partial documentation of classes prepared using Doxygen. Clearly, this is likely not to be enough since many details are left out from the Doxygen documentation. While the authors can see the need of a manual, it's still premature perhaps to devote effort on it.

We will update this site as much as possible inline with the preparation of more documentation. As a further source of information we suggest looking at some of the application code. Generally speaking, directories under \$YARP\_ROOT/src/experiments contain code using YARP for communication, image processing, etc. It might be a useful exercise to look at this code.