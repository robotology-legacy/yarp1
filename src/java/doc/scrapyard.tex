



\subsection{Sender's responsibilities}

\begin{itemize}

\item Sender creates initial connection using information 
from the name server.  The connection starts in TCP.

\item The sender needs to decide what the desired carrier
for the connection will be (tcp, udp, mcast).

\item If the connection will be associated with an output port, the
name of that port must be known (it will be communicated to the
recipient).

\end{itemize}

Phases are as follows:

\begin{itemize}

\item sendHeader
    
  \begin{itemize}

    \item sendProtocolSpecifier

      \begin{itemize}

      \item send 8 bytes: \packet{'Y' 'A' B1 B2 0 0 'R' 'P'}
      \item B1 and B2 identify the desired protocol
      (B1,B2) forms a little-endian integer whose value is
         7777 + K + 128*(requireAck),
      where K=0 for udp, K=1 for mcast, K=2 for shmem, K=3 for tcp.
      requireAck is an implementation choice controlling
      whether message receipt is acknowledge; see expectAck and sendAck
      phases.
    
      \item ALTERNATIVE: for text-mode, send 8 characters 'CONNECT '.
	
      \end{itemize}

    \item sendSenderSpecifier
      
      \begin{itemize}
      \item send 4 byte integer (little endian) giving length of port
      name plus one
      \item send port name in characters followed by null character
      \item ALTERNATIVE: for text-mode, send name (without null)
      followed by $\backslash$n character
      \end{itemize}

    \item sendExtraHeader      // carrier specific
      \begin{itemize}
	\item for tcp nothing more need be done
	\item for udp nothing more need be done
	\item for mcast, send 6 bytes.
         First 4 bytes specify a multicast IP address.
	 Next 2 bytes are a (bigendian - sorry) integer giving a port number.
         note that producing these numbers will require side communication
         with the name server.
      \end{itemize}
  \end{itemize}

  \item expectReplyToHeader    // carrier specific
    \begin{itemize}
      \item for tcp and udp, read 8 bytes:  \packet{'Y' 'A' B1 B2 0 0 'R' 'P'},
         where
	(B1,B2) make up a (little-endian) integer specifying a port number N.
	For udp, then switch over to udp network communication on that port.
      \item for mcast, switch over to mcast network communication on the IP and
	port number sent in sendExtraHeader
      \item ALTERNATIVE: for text-mode, expect nothing
    \end{itemize}

  \item sendIndex

    \begin{itemize}
    \item NOTE check isActive first: this is a carrier specific check
    - for mcast, a sender is ignored if it is redundant -- sending
    data to a multi-cast group that has already been sent.
    \item need to decide on the content to send - a set of LEN byte arrays.
    \item send 8 bytes: \packet{'Y' 'A' 10 0 0 0 'R' 'P'}
      This identifies the length of the index header as 10.
    \item send 10 bytes: \packet{LEN 1   255 255 255 255  255 255 255 255}
      This says there are LEN send blocks, 1 reply block expected,
      and that the sizes will be listed individually next (this
      format is for backward compatability with older yarp versions).
    \item A 4-byte little-endian integer is sent giving the length of
      each byte array.
    \item send 4-bytes: \packet{0 0 0 0} -- this asks for a reply length of 0.
      (this is actually an implementation choice, you can ask for a bigger
      reply if you like).
    \item ALTERNATIVE for text mode, send nothing.
    \end{itemize}

  \item sendContent

    \begin{itemize}
      \item NOTE check isActive first, as for sendIndex.
      \item send each of the LEN byte arrays in turn.
    \end{itemize}

  \item expectAck
    \begin{itemize}
    \item NOTE check requireAck and skip if appropiate.
    \item read 8 bytes: \packet{'Y' 'A' B1 B2 B3 B4 'R' 'P'}.
    \item B1-4 is a little-endian integer giving a length LEN.
    \item read LEN bytes.  This is the acknowledgement.
    \end{itemize}
\end{itemize}

\subsection{Receiver's Responsibilities}


\begin{itemize}
\item  start in TCP
\end{itemize}

Here are the phases:


\begin{itemize}
\item expectHeader
  
  \begin{itemize}
  \item expectProtocolSpecifier
    \begin{itemize}
    \item read 8 bytes: \packet{'Y' 'A' B1 B2 0 0 'R' 'P'}
    \item B1 and B2 identify the desired protocol (see sendProtocolSpecifier)
    \item ALTERNATIVE: for textmode, expect 'CONNECT ' -- only tcp supported
    \end{itemize}

  \item expectSenderSpecifier
    \begin{itemize}
    \item read 4 bytes -- a (little-endian) integer specifying a length.
    \item read that length of bytes, and interpret as a null-terminated
      string specifying the name of the sender port.
    \item ALTERNATIVE: for textmode, expect name (without null) followed by $\backslash$n
    \end{itemize}

  \item expectExtraHeader        // carrier specific
    \begin{itemize}
    \item for mcast, expect the content described in sendExtraHeader
    \end{itemize}
  \end{itemize}

\item respondToHeader
  \begin{itemize}
  \item send 8 bytes: \packet{'Y' 'A' B1 B2 B3 B4 'R' 'P'} where B1-4 is a 
    (little-endian) integer giving the port number of the receiver.
  \item for mcast and udp: switch to that network protocol at this time.
  \item ALTERNATIVE: for textmode, send nothing
  \end{itemize}

\item expectIndex (next phases repeat indefinitely)
  \begin{itemize}
  \item read 8 bytes: \packet{'Y' 'A' B1 B2 B3 B4 'R' 'P'} where B1-4 is a 
    (little-endian) integer of value 10.
  \item read 10 bytes: \packet{LEN 1 .....}
  \item read LEN 4-byte (little-endian) integers.  These specify
    the length of the series of LEN blocks of which the
    content is composed.  Sum these to get the TOTAL\_SIZE of the 
    message.
  \item read and discard 1 4-byte (little-endian) integer (this is
    the reply size).
  \item ALTERNATIVE: for text mode, expect nothing
  \end{itemize}

\item respondToIndex
  \begin{itemize}
  \item do nothing - the index is not individually acknowledged.
  \end{itemize}

\item expectBlock
  \begin{itemize}
  \item request a user determined number of blocks.  The user
    is expected to decode the content received.  The TOTAL\_SIZE
    of the message should be made available to help with this.
  \item ALT for text mode, read to newline instead
  \end{itemize}

\item sendAck
  \begin{itemize}
  \item read 8 bytes: \packet{'Y' 'A' B1 B2 B3 B4 'R' 'P'}
  \item B1-4 is a little-endian integer giving a length LEN (could be 0)
  \item send LEN bytes.  This is the acknowledgement.
  \end{itemize}
\end{itemize}



%% \subsection{YARP versioning}

%% Need to specify a way to know what version of yarp is available.

%% version 0 = Prior to versions existing.
%% Will switch to SHMEM regardless of other party.
%% Text-mode name server not supported.

%% Pending issues: 

%% How to deal with network-card choice in a transparent
%% way.  YARP1 - have user create a table of machine names and which
%% network-card they should use to communicate with which other machines
%% (by name).  YARP2?  

%% Each machine has a set of IPs, which apply to different networks.
%% For connection, need to find IPs that are mutually reachable.
%% How to discover this?

%% On registration, could identify a machine with its NIC 
%% MAC addresses (can't rely on having a good domain name present)
%% and give a list of IPs.  To determine connectivity, could go
%% through each NIC, see of any of target's IPs are reachable,
%% and if so if that is the correct machine.


%% Make MCAST do right thing with network-cards.

%% Make SHMEM inter-op work, either by handshaking or specifying 
%% capabilities in registration.

