
\documentclass[a4]{article}

\usepackage{graphicx}
\usepackage{fullpage}

\usepackage{code}
\CodeNoNumber

\newenvironment{codecase}[1]{\subsection{#1}}{}


\usepackage[sort]{natbib}
\newcommand{\citeasnoun}{\citet}
\renewcommand{\cite}{\citep}

%\emergencystretch=\hsize
\lefthyphenmin=2
\righthyphenmin=2
%\tolerance=9999

%\setcounter {topnumber}{7}
\renewcommand {\topfraction}{0.99}            % common default: 0.8
\renewcommand {\bottomfraction}{0.99}         % common default: 0.8
\setcounter   {totalnumber}{14}               % common default: 3
\renewcommand{\topfraction}{0.999}
\renewcommand {\textfraction}{0.01}           % common default: 0.2
\renewcommand {\floatpagefraction}{0.99}       % common default: 0.5

\newcommand{\dgrs}{$^{\circ}$}
\newcommand{\pflist}
  {     \renewcommand{\labelitemi}{$\triangleright$}
        \setlength{\itemsep}{0mm}
        \setlength{\parsep}{0mm}
        \setlength{\partopsep}{0mm}
        \setlength{\topsep}{0mm}
        \setlength{\parskip}{0mm}    }




\title{
YARP Protocol Specification
}


%
%  For one-to-one authur/affil correspondence
\author{Paul Fitzpatrick}


\begin{document}

\maketitle


\begin{abstract}

This document describes YARP protocol version 2.0.  At the time
of writing, the C++ version of YARP does not fully support this
protocol.  This specification describes a goal rather than an
implemented system.

\end{abstract}

\pagenumbering{arabic}



\section{Definition of terms}

The YARP library supports transmission of a stream of packets across
various protocols -- TCP, UDP, MCAST (multi-cast), shared memory, QNX
message passing -- insulating a user of the library from the
idiosyncratic details of the network technology used.  We call these
these low-level protocols the ``Carriers'', to distinguish them from
the higher-level protocols we will be concerned with here.

For the purposes of YARP, communication takes place throogh
``Connections'' between named entities called ``Ports''.
These form a directed graph, the ``YARP Network'', where Ports are the nodes,
and Connections are the edges.

Each Port is assigned a unique name, such as ``/motor/wheels/left''.  
Every Port is registered by name with
a ``name server''.  The goal is to ensure that if you know the name
of a Port, that is all you need in order to be able to 
communicate with it from any machine.

The purpose of Ports is to move ``Content'' (sequences of bytes) from
one thread to another (or several others) across process and machine
boundaries.  The flow of Content can be manipulated and monitored
externally (e.g. from the command-line) at run-time.  In other words,
the edges in the YARP Network are entirely mutable.

Ports are specialized into InputPorts and OutputPorts.
Any OutputPort can send Content to any number of InputPorts.  Any
InputPort can receive Content from any number of OutputPorts.  If an
OutputPort is configured to send Content to an InputPort, they are
said to have a Connection.  Connections can be freely added or 
removed, and may use different Carriers.



\section{The name server}

The name server is a program that listens on a known socket-port\footnote{We
write socket-ports to distinuish tcp/ip port numbers for sockets, 
from higher-level YARP Ports.}
on a known machine.
%
It tracks information about Ports in the YARP Network.
%
%
If you know the name
of a Port, you can query the name server to learn how to communicate
with that Port.

The name server maintains a set of records, whose key is a text
string.  The contents are at least hostname, socket number, and
protocol name.  This describes how to contact the port.  There is also
a description of what kinds of connections the port can or is willing
to participate in.  The set of protocols the port can accept an
incoming data connection for are named - this is the ``accept'' set.
The set of protocols the port can create an outgoing data connection
for are also named - this is the ``offer'' set.

For example, suppose you want to communicate with a Port called
``/write''.  The first step is to ask the name server about this
Port.  The name server runs on a known socket-port of a known machine,
listening for tcp connections.
It is usually queried through a library call, but for illustration
purposes we describe
querying it using telnet.  Suppose the name server is running on
machine 192.168.0.3 and listening on socket-port 10000 (we will
discuss a procedure for discovering this information later).
Then we can query the name server about the Port /write as follows:


\begin{code}
telnet 192.168.0.3 10000
\end{code}
%
The name server should start listening -- if the connection is refused,
something is wrong.  Once the connection is made, type:
%
\begin{code}
NAME_SERVER query /write
\end{code}
%
The server will respond with something of the form:
%
\begin{code}
registration name /write ip 5.255.112.227 port 10001 type tcp
*** end of message
\end{code}
%
So the Port named /write is listening on the machine with
IP address 5.255.112.227, on port 10001, and it expects TCP
connections.

How do Ports get registered in the same place?  Here's how to create a
(fake) registration
manually (usually it is of course done through a library call).
Telnet to the name server as before, and type:
%
\begin{code}
NAME_SERVER register /write
\end{code}
%
The server will respond with something of the form:
%
\begin{code}
registration name /write ip 5.255.112.227 port 10001 type tcp
*** end of message
\end{code}

The name server takes responsibility for allocating
socket-ports and identifying the machine the Port runs on.

Note that the protocol described here for communicating with the name
server is a YARP2 feature.  YARP1 used a different, binary protocol.
The human-readable protocol has been introduced to make the system
more transparent and easier to step through.

\begin{codecase}{NAME\_SERVER query PORT}
Requests registration information for the named port.  Response is of 
the following form:
\begin{code}
registration name PORT ip ADDRESS port NUMBER type CARRIER
*** end of message
\end{code}
For example:
\begin{code}
registration name /write ip 5.255.112.227 port 10001 type tcp
*** end of message
\end{code}
If there is no registration for the port, the registration line
is omitted, and instead the response is simply:
\begin{code}
*** end of message
\end{code}
\end{codecase}


\begin{codecase}{NAME\_SERVER register PORT}

Requests creation of registration information for the named port.  
Response is of the following form:
\begin{code}
registration name PORT ip ADDRESS port NUMBER type CARRIER
*** end of message
\end{code}
For example:
\begin{code}
registration name /write ip 5.255.112.227 port 10001 type tcp
*** end of message
\end{code}
%
Optionally, the user can take responsibility for more, and 
issue commands in one of the following forms:
\begin{code}
NAME_SERVER register PORT CARRIER
NAME_SERVER register PORT CARRIER IP
NAME_SERVER register PORT CARRIER IP NUMBER
\end{code}
Any value (including the port name) can be replaced by ``*'' to leave it 
up to the name-server to choose it.  For example:
\begin{code}
NAME_SERVER register * tcp 127.0.0.1 8080
\end{code}
Gives something of the form:
\begin{code}
registration name /tmp/port/1 ip 127.0.0.1 port 8080 type tcp
*** end of message
\end{code}
If you choose to set the ip yourself, be careful -- there is the 
possibility of problems with multiple ways to identify the same
machine.  It is best to let the name server choose a name,
which it should do in a consistent way.  If a machine has
multiple ip addresses on multiple networks, that can be 
handled -- see the 
discussion of the {\tt ips} property in the section on {\tt set}.
That is important for the purposes of controlling which 
network is used for connections from one port to another.


\end{codecase}

\begin{codecase}{NAME\_SERVER set PORT PROPERTY VALUE1 VALUE2 \ldots}

The name server can store extra properties of a port, beyond the
bare details associated with registration.  The {\tt set} command
is used to do this.  For example, the command:
\begin{code}
NAME_SERVER set /write offers tcp udp mcast
\end{code}
Gets the following response:
\begin{code}
port /write property offers = tcp udp mcast
\end{code}
The {\tt get} and {\tt check} commands can then be used to query
such properties.

There are some special properties used by YARP.  Property ``ips''
can list multiple identifiers of a machine.  Property ``offers''
lists carriers that an output port can support.  Propery ``accepts''
lists carriers that an input port can support.

\end{codecase}

\begin{codecase}{NAME\_SERVER get PORT PROPERTY}
Gets the values of a stored property. For example, 
after the {\tt set} command example shown earlier, the command:
\begin{code}
NAME_SERVER get /write offers
\end{code}
Returns the following response:
\begin{code}
port /write property offers = tcp udp mcast
\end{code}
\end{codecase}

\begin{codecase}{NAME\_SERVER check PORT PROPERTY VALUE}
Not yet documented.
\end{codecase}

\begin{codecase}{NAME\_SERVER match PORT PROPERTY VALUE}
Not yet documented.
\end{codecase}

\begin{codecase}{NAME\_SERVER connect}
Not yet documented.
\end{codecase}

\begin{codecase}{quit}
Not yet documented.
\end{codecase}


\section{The standard yarp utility}

We specify a standard utility called ``yarp'' for performing a set of
useful operations with a YARP Network.  We now give some ways to run that 
utility.  For any of the examples below, the word ``verbose'' can 
be inserted as the first argument to increase the number of 
messages given.

\subsection{yarp}

This should list a summary of the usages mentioned in this section.

\subsection{yarp server}

This will start up a name server on the current machine.

\subsection{yarp where}

This will report where the name server is believed to be running.

\subsection{yarp version}

This will report on the yarp version available.

\subsection{yarp name COMMAND ARG1 ARG2 \ldots}

This will send the given command and arguments to the name
server and report the results.  Equivalent to the telnet
examples given earlier, except the ``NAME\_SERVER'' prefix
is added automatically to the message send to the name server.

\subsection{yarp read PORT}

This will create an input port of the specified name.  The port
reads ``yarp bottles'' (specified later) and prints their content
to standard output.

\subsection{yarp write PORT}

This will create an output port of the specified name.  The port
reads from standard input and writes ``yarp bottles'' (specified later).
Optionally, a list of input ports to connect to automatically can be
appended to the command.

\subsection{yarp connect OUTPUT\_PORT INPUT\_PORT}

This will request the specified output port to send its output in 
future to the specified input port.

Optionally, the protocol to be used can be added as an extra argument
(e.g. tcp, udp, mcast, ...).

\subsection{yarp disconnect OUTPUT\_PORT INPUT\_PORT}

This will request the specified output port to cease sending its output to
the specified input port.



\section{Communication with Ports}

Every Port is always available for general communication -- to request
that new connections be created, old connections be removed, to
inquire after status and health, etc.  That communication can take
place in binary or (in YARP2) text form.  We describe here text-mode
operation for simplicity.

Suppose we have created ports as follows by typing the following in
different terminals:
\begin{code}
  yarp server
  yarp write /write
  yarp read /read
  yarp read /read2
\end{code}
%
Now we can connect and disconnect things as follows:
\begin{code}
  yarp connect /write /read
  yarp disconnect /write /read
  yarp connect /write /read mcast
  yarp connect /write /read2
\end{code}

Here's how we could do these connections manually:

\begin{code}
command:  yarp where
response: Name server is available at ip 192.168.0.3 port 10000

command:  telnet 192.168.0.3 10000
type:     NAME_SERVER query /write
response: registration name /write ip 192.168.0.3 port 10001 type tcp
          *** end of message
          [connection closes]

command:  telnet 192.168.0.3 10001
type:     CONNECT anonymous
response: Welcome anonymous
type:     *
response: This is /write
          There are no outgoing connections
          There is this connection from anonymous to /write using protocol tcp
          *** end of message
type:     /read
response: Connected to /read
type:     *
response: This is /write
          There is a connection from /write to /read using protocol tcp
          There is this connection from anonymous to /write using protocol tcp
          *** end of message
type:     !/read
response: Removing connection from /write to /read
type:     /mcast://read
response: Connected to /read
type:     /read2
response: Connected to /read2
type:     *
response: This is /write
          There is a connection from /write to /read using protocol mcast
          There is a connection from /write to /read2 using protocol tcp
          There is this connection from anonymous to /write using protocol tcp
          *** end of message
type:     q
response: Bye bye
          [connection closes]

command:  telnet 192.168.0.3 10000
type:     NAME_SERVER query /read
response: registration name /write ip 192.168.0.3 port 10002 type tcp
          *** end of message
          [connection closes]

command:  telnet 192.168.0.3 10002
type:     CONNECT anonymous
response: Welcome anonymous
type:     *
response: This is /read
          There are no outgoing connections
          There is a connection from /write to /read using protocol mcast
          There is this connection from anonymous to /read using protocol tcp
          *** end of message
type:     q
response: Bye bye
          [connection closes]

\end{code}


The protocol used for communicating with a port is layed on top of the
protocol described in the next section.  The ``user data'' is 
as follows:

\begin{itemize}

\item In text-mode, we send an 8 byte header: 0,0,0,0, '~', CHAR, 0, 1.

\item CHAR is a character that identifies what the message is about.

  \begin{itemize}
    
  \item CHAR = 'd': this header is used to signal that content data is
  arriving next.
    
  \item CHAR = 0: this header is used to signal that a control message
  is following.

  \end{itemize}

\item for the control message case (CHAR = 0) the remainder of the message 
  is interpreted as a string S.

  \begin{itemize}
   \item S begins with '/', e.g. '/read': this is a request to add a
   Connection to the named InputPort.

   \item S begins with '!', e.g. '!/read': this is a request to remove
   a Connection to the named InputPort.

   \item S begins with '~', e.g. '~/read': this is a request to remove
   a Connection from the named OutputPort.

   \item S is '*': this is a request for the port to dump information.

   \item S is 'q': specific connection that message is received on
   should now shut down.
  \end{itemize}

\end{itemize}




\subsection{Communication between Ports}

This is the protocol for a single connection to an input port.
Such connections can be made at any time.  At creation,
the properties of a connection are:

\begin{itemize}

\item An address -- the machine name and socket-port at which
the input port is listening.

\item An output port name -- typically connections are made
between output ports and input ports.  The output port name
needs to be retained for disconnection and housekeeping.
If the connection is not actually associated with a port,
the name can be anything that starts with something
other than ``/''.

\item The input port name -- the name of the target.

\end{itemize}


The connection protocol has several phases.

\subsection{Sender's responsibilities}

\begin{itemize}

\item Sender creates initial connection using information 
from the name server.  The connection starts in TCP.

\item The sender needs to decide what the desired carrier
for the connection will be (tcp, udp, mcast).

\item If the connection will be associated with an output port, the
name of that port must be known (it will be communicated to the
recipient).

\end{itemize}

Phases are as follows:

\begin{itemize}

\item sendHeader
    
  \begin{itemize}

    \item sendProtocolSpecifier

      \begin{itemize}

      \item send 8 bytes: 'Y' 'A' B1 B2 0 0 'R' 'P'
      \item B1 and B2 identify the desired protocol
         P=tcp   B1=0x64 B2=0x1e
         P=udp   B1=0x61 B2=0x1e
         P=mcast B1=0x62 B2=0x1e
      More correctly, {B1,B2} forms a little-endian integer whose value is
         7777 + K + 128*(requireAck)
      where 
         P=tcp   K=3
	 P=udp   K=0
	 P=mcast K=1
      requireAck is an implementation choice; see expectAck and sendAck.
      If it is true, message receipt must be acknowledged.
    
      \item ALT: for textmode, send 'CONNECT ' -- only tcp supported
	
      \end{itemize}

    \item sendSenderSpecifier
      
      \begin{itemize}
      \item send 4 byte integer (little endian) giving length of port name plus one
      \item send port name in characters followed by null character
      \item ALT: for textmode, send name (without null) followed by \\n
      \end{itemize}

    \item sendExtraHeader      // carrier specific
      \begin{itemize}
	\item for tcp nothing more need be done
	\item for udp nothing more need be done
	\item for mcast, send 6 bytes
         first 4 bytes specify a multicast IP address.
	 next 2 bytes are a (bigendian - sorry) integer giving a port number.
         note that producing these numbers will require side communication
         with the name server.
      \end{itemize}
  \end{itemize}

  \item expectReplyToHeader    // carrier specific
    \begin{itemize}
      \item for tcp and udp, read 8 bytes:  'Y' 'A' B1 B2 0 0 'R' 'P' where
	B1 B2 make up a (little-endian) integer specifying a port number N.
	For udp, then switch over to udp network communication on that port.
      \item for mcast, switch over to mcast network communication on the IP and
	port number sent in sendExtraHeader
      \item ALT: for textmode, expect nothing
    \end{itemize}

  \item sendIndex

    \begin{itemize}
    \item NOTE check isActive (carrier specific) first: for mcast, a sender is ignored if it is redundant -- sending
    data to a multi-cast group that has already been sent.

    \item need to decide on the content to send - a set of LEN byte arrays.
    \item send 8 bytes: 'Y' 'A' 10 0 0 0 'R' 'P'
      This identifies the length of the index header as 10.
    \item send 10 bytes: LEN 1   255 255 255 255  255 255 255 255
      This says there are LEN send blocks, 1 reply block expected,
      and that the sizes will be listed individually next.
    \item A 4-byte little-endian integer is sent giving the length of
      each byte array.
    \item send 4-bytes: 0 0 0 0 -- this asks for a reply length of 0.
    \item ALT for text mode, send nothing.
    \end{itemize}

  \item sendContent

    \begin{itemize}
      \item NOTE check isActive first
      \item send each of the LEN byte arrays in turn.
    \end{itemize}

  \item expectAck
    \begin{itemize}
    \item NOTE check requireAck
    \item read 8 bytes: 'Y' 'A' B1 B2 B3 B4 'R' 'P'
    \item B1-4 is a little-endian integer giving a length LEN
    item read LEN bytes.  This is the acknowledgement.
    \end{itemize}
\end{itemize}

\subsection{Receiver's Responsibilities}


\begin{itemize}
\item  start in TCP
\end{itemize}

Here are the phases:


\begin{itemize}
\item expectHeader
  
  \begin{itemize}
  \item expectProtocolSpecifier
    \begin{itemize}
    \item read 8 bytes: 'Y' 'A' B1 B2 0 0 'R' 'P'
    \item B1 and B2 identify the desired protocol (see sendProtocolSpecifier)
    \item ALT: for textmode, expect 'YAtextRP' -- only tcp supported
    \end{itemize}

  \item expectSenderSpecifier
    \begin{itemize}
    \item read 4 bytes -- a (little-endian) integer specifying a length.
    \item read that length of bytes, and interpret as a null-terminated
      string specifying the name of the sender port.
    \item ALT: for textmode, expect name (without null) followed by $\backslash$n
    \end{itemize}

  \item expectExtraHeader        // carrier specific
    \begin{itemize}
    \item for mcast, expect the content described in sendExtraHeader
    \end{itemize}
  \end{itemize}

\item respondToHeader
  \begin{itemize}
  \item send 8 bytes: 'Y' 'A' B1 B2 B3 B4 'R' 'P' where B1-4 is a 
    (little-endian) integer giving the port number of the receiver.
  \item for mcast and udp: switch to that network protocol at this time.
  \item ALT: for textmode, send nothing
  \end{itemize}

\item expectIndex (next phases repeat indefinitely)
  \begin{itemize}
  \item read 8 bytes: 'Y' 'A' B1 B2 B3 B4 'R' 'P' where B1-4 is a 
    (little-endian) integer of value 10.
  \item read 10 bytes: LEN 1 .....
  \item read LEN 4-byte (little-endian) integers.  These specify
    the length of the series of LEN blocks of which the
    content is composed.  Sum these to get the TOTAL\_SIZE of the 
    message.
  \item read and discard 1 4-byte (little-endian) integer (this is
    the reply size).
  \item ALT for text mode, expect nothing
  \end{itemize}

\item respondToIndex
  \begin{itemize}
  \item do nothing
  \end{itemize}

\item expectBlock
  \begin{itemize}
  \item request a user determined number of blocks.  The user
    is expected to decode the content received.  The TOTAL\_SIZE
    of the message should be made available to help with this.
  \item ALT for text mode, read to newline instead
  \end{itemize}

\item sendAck
  \begin{itemize}
  \item read 8 bytes: 'Y' 'A' B1 B2 B3 B4 'R' 'P'
  \item B1-4 is a little-endian integer giving a length LEN (could be 0)
  \item send LEN bytes.  This is the acknowledgement.
  \end{itemize}
\end{itemize}



%% \subsection{YARP versioning}

%% Need to specify a way to know what version of yarp is available.

%% version 0 = Prior to versions existing.
%% Will switch to SHMEM regardless of other party.
%% Text-mode name server not supported.

%% Pending issues: 

%% How to deal with network-card choice in a transparent
%% way.  YARP1 - have user create a table of machine names and which
%% network-card they should use to communicate with which other machines
%% (by name).  YARP2?  

%% Each machine has a set of IPs, which apply to different networks.
%% For connection, need to find IPs that are mutually reachable.
%% How to discover this?

%% On registration, could identify a machine with its NIC 
%% MAC addresses (can't rely on having a good domain name present)
%% and give a list of IPs.  To determine connectivity, could go
%% through each NIC, see of any of target's IPs are reachable,
%% and if so if that is the correct machine.


%% Make MCAST do right thing with network-cards.

%% Make SHMEM inter-op work, either by handshaking or specifying 
%% capabilities in registration.


\subsection{YARP URIs}

Port names in YARP can contain multiple special elements.
We've seen names such as ``/write''.  We can also have
names such as ``udp://write'' which means ``connect
to the port named /write using the udp carrier''.

We can also prepend a network selector of the form
``/net=NETNAME/''.  For example, a name such as
``udp://net=196/write'' means ``connect to the 
port named /write using the udp carrier, and make the
connection on the network with ip addresses beginning with
196''.
%
This is useful in scenarios with multiple networks, where it may be
desirable to route connections through particular networks (for
example, to devote a network to time-critical traffic).  This
functionality is supported primarily with the help of the name server.
The ip it reports for a machine is usually a reasonable default, but
the user can choose using ``net='' to request a name on a particular
network.

Symbolic network names can be configured.  This process 
is not yet documented.

%% How should multiple networks work?  Currently in YARP1, a port is
%% associated with a particular network.  If two ports are on different
%% networks, they can't communicate, even if there is connectivity
%% (I think).

%% It would seem more desirable to associate the choice of network,
%% if there is a choice, with the edges (connections) and not the
%% nodes (ports).

%% Let's require that all nodes can reach the name server and vice
%% versa.

%% Single-hop network choice for tcp and udp can be made by selecting one
%% of a choice of IPs for the target.  Could add a connection attribute
%% ``over 192.*'' as a way to choose net?  Or text names.

%% Done!


\subsection{Carriers supported}

An implementation of YARP2 must support at least the ``tcp''
and ``text'' protocol.  Other carriers that may be supported:
``udp'', ``mcast'', ``shmem''.

\end{document}

