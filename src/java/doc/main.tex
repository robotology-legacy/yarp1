
\section{YARP communications protocol}

YARP has good support for transmission of a stream of packets
across various network protocols -- TCP, UDP, MCAST (multi-cast),
and shared memory.  We call these the ``carriers''.

\subsection{The logical view}

A YARP network is made up of Port objects.  Each Port object has
a name, such as ``/motor/wheels/left''.  The names of all Port
objects are registered with a name server.  If you know the name
of a Port, that is all you need in order to be able to 
communicate with it from any machine.

The purpose of Ports is to move ``Content'' (sequences of bytes) from
one thread to another (or several others) across process and machine
boundaries.  The flow of Content can be manipulated and monitored
externally (e.g. from the command-line) at run-time.

Ports are specialized into InputPorts and OutputPorts.
Any OutputPort can send content to any number of InputPorts.  Any
InputPort can receive content from any number of OutputPorts.  If an
OutputPort is configured to send content to an InputPort, they are
said to have a Connection.  Connections can be freely added or 
removed, and may use different network protocols (TCP, UDP, Multicast).




\subsection{The name server}

If you know the name
of a Port, you can query the name server for a way to communicate
with that Port.
For example, suppose you want to communicate with a Port called
``write''.  The first step is to ask the name server about this
Port.  The name server runs on a known port-number of a known machine.
It is usually queried through a library call, but can also be
queried using telnet.  For example:

\begin{verbatim}
run:
  telnet <name-server-machine> <name-server-port-number>

type:
  NAME_SERVER query /write

server responds with something like:
  ip 192.168.0.3 port 10001 type tcp
  *** end of message
\end{verbatim}

So the Port named ``/write'' is listening on the machine with
IP address 192.168.0.3, on port 10001, and it expects TCP
connections.

How do Ports get registered in the same place?  Here's how
to do it manually (usually it is done through a library call).

\begin{verbatim}
run:
  telnet <name-server-machine> <name-server-port-number>

type:
  NAME_SERVER register /write

server responds with something like:
  ip 192.168.0.3 port 10001 type tcp
  *** end of message
\end{verbatim}

Note that the name server takes responsibility for allocating
port-numbers.

The use of telnet with the name server is a YARP2 feature.

We can also ``log on'' to the ports themselves (YARP2 feature).

\begin{verbatim}

Assume we have typed the following in different terminals:
  yarp server
  yarp write /write
  yarp read /read
  yarp read /read2

Now we can connect and disconnect things as follows:
  yarp connect /write /read
  yarp disconnect /write /read
  yarp connect /write /read mcast
  yarp connect /write /read2

We could also do this more manually as follows.

command:  yarp where
response: Name server is available at ip 192.168.0.3 port 10000


command:  telnet 192.168.0.3 10000
type:     NAME_SERVER query /write
response: ip 192.168.0.3 port 10001 type tcp
          *** end of message
          [connection closes]


command:  telnet 192.168.0.3 10001
type:     CONNECT anonymous
response: Welcome anonymous
type:     *
response: This is /write
          There are no outgoing connections
          There is this connection from anonymous to /write using protocol tcp
          *** end of message
type:     /read
response: Connected to /read
type:     *
response: This is /write
          There is a connection from /write to /read using protocol tcp
          There is this connection from anonymous to /write using protocol tcp
          *** end of message
type:     !/read
response: Removing connection from /write to /read
type:     /mcast://read
response: Connected to /read
type:     /read2
response: Connected to /read2
type:     *
response: This is /write
          There is a connection from /write to /read using protocol mcast
          There is a connection from /write to /read2 using protocol tcp
          There is this connection from anonymous to /write using protocol tcp
          *** end of message
type:     q
response: Bye bye
          [connection closes]


command:  telnet telnet 192.168.0.3 10000
type:     NAME_SERVER query /read
response: ip 192.168.0.3 port 10002 type tcp
          *** end of message
          [connection closes]


command:  telnet 192.168.0.3 10002
type:     CONNECT anonymous
response: Welcome anonymous
type:     *
response: This is /read
          There are no outgoing connections
          There is a connection from /write to /read using protocol mcast
          There is this connection from anonymous to /read using protocol tcp
          *** end of message
type:     q
response: Bye bye
          [connection closes]

\end{verbatim}


\subsection{connection protocol}

This is the protocol for a single connection to an input port.
Such connections can be made at any time.  At creation,
the properties of a connection are:

\begin{itemize}

\item An address -- the machine name and socket number at which
the input port is listening.

\item An output port name -- typically connections are made
between output ports and input ports.  The output port name
needs to be retained for disconnection and housekeeping.

\item The input port name -- the name of the target.

\end{itemize}

\begin{verbatim}

SENDER:

  * start in TCP
  * need to know desired protocol P = one of tcp,udp,mcast
  * need to know name of associated port, e.g. ``/writer''

  sendHeader

    sendProtocolSpecifier
    * send 8 bytes: 'Y' 'A' B1 B2 0 0 'R' 'P'
    * B1 and B2 identify the desired protocol
         P=tcp   B1=0x64 B2=0x1e
         P=udp   B1=0x61 B2=0x1e
         P=mcast B1=0x62 B2=0x1e
      More correctly, {B1,B2} forms a little-endian integer whose value is
         7777 + K + 128*(requireAck)
      where 
         P=tcp   K=3
	 P=udp   K=0
	 P=mcast K=1
      requireAck is an implementation choice; see expectAck and sendAck.
      If it is true, message receipt must be acknowledged.
    * ALT: for textmode, send 'CONNECT ' -- only tcp supported

    sendSenderSpecifier
    * send 4 byte integer (little endian) giving length of port name plus one
    * send port name in characters followed by null character
    * ALT: for textmode, send name (without null) followed by \\n

    sendExtraHeader      // carrier specific
    * for tcp nothing more need be done
    * for udp nothing more need be done
    * for mcast, send 6 bytes
         first 4 bytes specify a multicast IP address.
	 next 2 bytes are a (bigendian - sorry) integer giving a port number.
         note that producing these numbers will require side communication
         with the name server.

  expectReplyToHeader    // carrier specific
    * for tcp and udp, read 8 bytes:  'Y' 'A' B1 B2 0 0 'R' 'P' where
      B1 B2 make up a (little-endian) integer specifying a port number N.
      For udp, then switch over to udp network communication on that port.
    * for mcast, switch over to mcast network communication on the IP and
      port number sent in sendExtraHeader
    * ALT: for textmode, expect nothing

  if (isActive)          // carrier specific
  * for mcast, a sender is ignored if it is redundant -- sending
    data to a multi-cast group that has already been sent.
  * in all other cases, isActive is true

    sendIndex
    * need to decide on the content to send - a set of LEN byte arrays.
    * send 8 bytes: 'Y' 'A' 10 0 0 0 'R' 'P'
      This identifies the length of the index header as 10.
    * send 10 bytes: LEN 1   255 255 255 255  255 255 255 255
      This says there are LEN send blocks, 1 reply block expected,
      and that the sizes will be listed individually next.
    * A 4-byte little-endian integer is sent giving the length of
      each byte array.
    * send 4-bytes: 0 0 0 0 -- this asks for a reply length of 0.
    * ALT for text mode, send nothing.

    sendContent
    * send each of the LEN byte arrays in turn.

  if (requireAck)

    expectAck
    * read 8 bytes: 'Y' 'A' B1 B2 B3 B4 'R' 'P'
    * B1-4 is a little-endian integer giving a length LEN
    * read LEN bytes.  This is the acknowledgement.

RECEIVER:

  * start in TCP

  expectHeader

    expectProtocolSpecifier
    * read 8 bytes: 'Y' 'A' B1 B2 0 0 'R' 'P'
    * B1 and B2 identify the desired protocol (see sendProtocolSpecifier)
    * ALT: for textmode, expect 'YAtextRP' -- only tcp supported

    expectSenderSpecifier
    * read 4 bytes -- a (little-endian) integer specifying a length.
    * read that length of bytes, and interpret as a null-terminated
      string specifying the name of the sender port.
    * ALT: for textmode, expect name (without null) followed by \\n

    expectExtraHeader        // carrier specific
    * for mcast, expect the content described in sendExtraHeader

  respondToHeader
  * send 8 bytes: 'Y' 'A' B1 B2 B3 B4 'R' 'P' where B1-4 is a 
    (little-endian) integer giving the port number of the receiver.
  * for mcast and udp: switch to that network protocol at this time.
  * ALT: for textmode, send nothing

  repeat

    expectIndex
    * read 8 bytes: 'Y' 'A' B1 B2 B3 B4 'R' 'P' where B1-4 is a 
    (little-endian) integer of value 10.
    * read 10 bytes: LEN 1 .....
    * read LEN 4-byte (little-endian) integers.  These specify
      the length of the series of LEN blocks of which the
      content is composed.  Sum these to get the TOTAL_SIZE of the 
      message.
    * read and discard 1 4-byte (little-endian) integer (this is
      the reply size).
    * ALT for text mode, expect nothing

    respondToIndex
    * do nothing

    repeat based on user code
    
      expectBlock
      * request a user determined number of blocks.  The user
        is expected to decode the content received.  The TOTAL_SIZE
	of the message should be made available to help with this.
      * ALT for text mode, read to newline instead

    sendAck
    * read 8 bytes: 'Y' 'A' B1 B2 B3 B4 'R' 'P'
    * B1-4 is a little-endian integer giving a length LEN (could be 0)
    * send LEN bytes.  This is the acknowledgement.


\end{verbatim}


\subsection{port protocol}

This is the protocol used for communicating with a port -- 
starting, stopping connections.

\begin{verbatim}
Send 8 byte header: 0,0,0,0, '~', CHAR, 0, 1
where CHAR is a character that identifies what the message is about.

CHAR = 'd': this header is used to signal that content data is arriving next.

CHAR = 0: this header is used to signal that a control message is following.

  For this case, the remainder of the message is interpreted as a string S.

   S begins with '/', e.g. '/read': this is a request to add a Connection 
   to the named InputPort.

   S begins with '!', e.g. '!/read': this is a request to remove a Connection 
   to the named InputPort.

   S begins with '~', e.g. '~/read': this is a request to remove a Connection 
   from the named OutputPort.

   S is '*': this is a request for the port to dump information.

   S is 'q': specific connection that message is received on should 
   now shut down.

\end{verbatim}



\subsection{name protocol}

This is the protocol used for communication with a name server.

The name server maintains a set of records, whose key is a text
string.  The contents are a hostname, socket number, and protocol
name.  This describes how to contact the port.  There is also
a description of what kinds of connections the port can or is
willing to participate in.  The set of protocols the port can
accept an incoming data connection for are named - this is the 
``accept'' set.
The set of protocols the port can create an outgoing data 
connection for are also named - this is the ``offer'' set.

\subsection{data protocol}

Effectively user-level, not up to the YARP communications library.


\subsection{YARP versioning}


version 0 = Prior to versions existing.
Will switch to SHMEM regardless of other party.
Text-mode name server not supported.

Pending issues: 

How to deal with network-card choice in a transparent
way.  YARP1 - have user create a table of machine names and which
network-card they should use to communicate with which other machines
(by name).  YARP2?  

Each machine has a set of IPs, which apply to different networks.
For connection, need to find IPs that are mutually reachable.
How to discover this?

On registration, could identify a machine with its NIC 
MAC addresses (can't rely on having a good domain name present)
and give a list of IPs.  To determine connectivity, could go
through each NIC, see of any of target's IPs are reachable,
and if so if that is the correct machine.


Make MCAST do right thing with network-cards.

Make SHMEM inter-op work, either by handshaking or specifying 
capabilities in registration.

